\documentclass[12pt]{article}
\usepackage[top=1in, bottom=1in, left=1in, right=1in]{geometry}

\title{NE 155, Class 3, S14 \\
Types of Equations in the Engineering Fields}
\author{Rachel Slaybaugh}
\date{January 27, 2014}

\usepackage{setspace}
\onehalfspacing

\usepackage{amsmath}
\newcommand{\nth}{n\ensuremath{^{\text{th}}} }
%\newcommand{\1st}{\ensuremath{1^{\text{st}}} }
%\newcommand{\3rd}{\ensuremath{3^{\text{rd}}} }


\begin{document}
\begin{center}
{\bf NE 155, Class 3, S14 \\
Types of Equations in the Engineering Fields \\ January 27, 2014}
\end{center}

\setlength{\unitlength}{1in}
\begin{picture}(6,.1) 
\put(0,0) {\line(1,0){6.25}}         
\end{picture}


\noindent \textbf{Introduction}

In science and engineering in general, and nuclear engineering and reactor analysis in specific, we encounter a wide range of mathematical physics equations. In today's lecture we will introduce some of them.

\begin{itemize}
\item Ordinary differential equations (ODEs)
\item Partial differential equations (PDEs)
  \begin{itemize}
  \item Elliptic PDEs
  \item Parabolic PDEs
  \item Hyperbolic PDEs
  \end{itemize}
\item Integro-differential equations
\item Integral equations
\end{itemize}

\section{ODEs}

The most general form of an \nth order linear ordinary differential eqn.\ is
%
\begin{equation}
a_{n+1}(x)y^{(n)} + a_{n}(x)y^{(n-1)} + \cdots + a_{2}(x)y^{(1)} + a_{1}(x)y + a_0 = f(x) \nonumber
\end{equation}
%
\noindent where
\begin{itemize}
\item $a_n$ are coefficients
\item $y^{(n)}$ is the \nth derivative of $y$.
\end{itemize}

Boundary conditions:
\begin{enumerate}
\item if y and its derivatives are given at one end of the domain/interval (time 0 if there's time or x 0 if there's only space, etc.): Initial Value Problem (IVP)
\item if y and/or its derivatives are given at \underline{each} end of the interval: Boundary Value Problem (BVP)
\end{enumerate}

\noindent \textbf{Linear 1st order ODE's}

\underline{Reminders}
\begin{itemize}
\item Linear means each term is either a constant or the product of a constant and the first power of a single variable (in this case y).
\item the highest derivative is to the first power.
\end{itemize}

\noindent \underline{Example}:
\begin{equation}
y' + 3y = sin(x) \nonumber
\end{equation}
%
\begin{itemize}
\item IVP if boundary conditions are y(0) = 1; y'(0) = 2
\item BVP if boundary conditions are y(0)=-1, y(1) = 3 on the domain [0,1]
\end{itemize}
%
In this case the general solution is obtained through the use of an integrating factor.

\noindent \underline{Example}:

Point Kinetics analysis of a nuclear reactor is an IVP, linear 1st order ODE.
%
\begin{align}
\frac{dn(t)}{dt} &= \frac{\rho(t) - \beta}{l^*}n(t) + \sum_{i=1}^{N} \lambda_i C_i(t) \nonumber \\
%
\frac{dC_i(t)}{dt} &= \frac{\beta_i}{l^*}n(t) - \lambda_i C_i(t) \qquad i=1,\dots,N \nonumber
\end{align}
%
Where
%
\begin{itemize}
\item n = \# neutrons / s
\item $\beta$ = fraction of delayed neutrons
\item $\lambda_i$ = effective decay constant of the ith precursor
\item $C_i(t)$ = delayed neutron concentration of the ith precursor
\item $l^*$ = mean neutron lifetime
\item $\rho = \frac{k-1}{k}$ = reactivity
\end{itemize}
%
BCs: $n(t=0) = n_0$ and $C_i(t=0) = C_{i,0}$ for $i=1,\dots,N$.





\end{document}
