%--------------------------------------------------------------------
% NE 155 (intro to numerical simulation of radiation transport)
% Spring 2014

% formatting
\documentclass[12pt]{article}
\usepackage[top=1in, bottom=1in, left=1in, right=1in]{geometry}

\usepackage{setspace}
\onehalfspacing

\setlength{\parindent}{0mm} \setlength{\parskip}{1em}


% packages
\usepackage{amssymb}
%% The amsthm package provides extended theorem environments
\usepackage{amsthm}
\usepackage{epsfig}
\usepackage{times}
\renewcommand{\ttdefault}{cmtt}
\usepackage{amsmath}
\usepackage{graphicx} % for graphics files

% Draw figures yourself
\usepackage{tikz} 

% The float package HAS to load before hyperref
\usepackage{float} % for psuedocode formatting
\usepackage{xspace}

% from Denovo methods manual
\usepackage{mathrsfs}
\usepackage[mathcal]{euscript}
\usepackage{color}
\usepackage{array}

\usepackage[pdftex]{hyperref}

\newcommand{\nth}{n\ensuremath{^{\text{th}}} }
\newcommand{\ve}[1]{\ensuremath{\mathbf{#1}}}
\newcommand{\macro}{\ensuremath{\Sigma}}
\newcommand{\vOmega}{\ensuremath{\hat{\Omega}}}

\newcommand{\cc}[1]{\ensuremath{\overline{#1}}}
\newcommand{\ccm}[1]{\ensuremath{\overline{\mathbf{#1}}}}


%--------------------------------------------------------------------
%--------------------------------------------------------------------
\begin{document}
\begin{center}
{\bf NE 155, midterm 2 review S14 \\
April 25, 2014}
\end{center}

\setlength{\unitlength}{1in}
\begin{picture}(6,.1) 
\put(0,0) {\line(1,0){6.25}}         
\end{picture}

%--------------------------------------------------------------------
Here are the topics we've covered and that are fair game for the exam:

\underline{More deterministic methods}
\begin{itemize}
\item Finite difference derivation; application to the 1-D, fixed source diffusion equation
\item Finite volume method
  \begin{itemize}
  \item Derivation
  \item Application to the diffusion equation
  \item Vacuum and reflecting boundary conditions
  \item Simplification to homogeneous, uniform mesh
  \end{itemize}
  
\item Methods for solving the system of equations we created
  \begin{itemize}
  \item Directly with Thomas Algorithm
  \item Iteratively with Jacobi, GS, or SOR
  \end{itemize}

\item Eigenvalue form of the 1-D DE
  \begin{itemize}
  \item form of the equation; applying finite difference
  \item applying finite volume method, including BCs
  \end{itemize}
  
\item Solving the Eigenvalue equations
  \begin{itemize}
  \item determining convergence of $k$ and $\phi$
  \item calculating $k$
  \item Power Iteration
  \end{itemize}
\end{itemize}


\underline{Monte Carlo methods}
\begin{itemize}
\item overview/history of MC

\item random number generators
  \begin{itemize}
  \item 
  \end{itemize}

\item random sampling
  \begin{itemize}
  \item 
  \end{itemize}

\item statistics
  \begin{itemize}
  \item 
  \end{itemize}

\item transport algorithm
  \begin{itemize}
  \item 
  \end{itemize}
  
\item tallies
  \begin{itemize}
  \item 
  \end{itemize}

\end{itemize}

The exam will be 50 minutes long and closed book. You may use a calculator.

\end{document}