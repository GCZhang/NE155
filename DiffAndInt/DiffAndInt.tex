%--------------------------------------------------------------------
% NE 155 (intro to numerical simulation of radiation transport)
% Spring 2014

% formatting
\documentclass[12pt]{article}
\usepackage[top=1in, bottom=1in, left=1in, right=1in]{geometry}

\usepackage{setspace}
\onehalfspacing

\setlength{\parindent}{0mm} \setlength{\parskip}{1em}


% packages
\usepackage{amssymb}
%% The amsthm package provides extended theorem environments
\usepackage{amsthm}
\usepackage{epsfig}
\usepackage{times}
\renewcommand{\ttdefault}{cmtt}
\usepackage{amsmath}
\usepackage{graphicx} % for graphics files

% Draw figures yourself
\usepackage{tikz} 

% The float package HAS to load before hyperref
\usepackage{float} % for psuedocode formatting
\usepackage{xspace}

% from Denovo methods manual
\usepackage{mathrsfs}
\usepackage[mathcal]{euscript}
\usepackage{color}
\usepackage{array}

\usepackage[pdftex]{hyperref}

\newcommand{\nth}{n\ensuremath{^{\text{th}}} }
\newcommand{\ve}[1]{\ensuremath{\mathbf{#1}}}
\newcommand{\macro}{\ensuremath{\Sigma}}
\newcommand{\vOmega}{\ensuremath{\hat{\Omega}}}

\newcommand{\cc}[1]{\ensuremath{\overline{#1}}}
\newcommand{\ccm}[1]{\ensuremath{\overline{\mathbf{#1}}}}


%--------------------------------------------------------------------
%--------------------------------------------------------------------
\begin{document}
\begin{center}
{\bf NE 155, Classes 13 \& 14, S14 \\
Differentiation \& Integration \\ February 24 and 26, 2014}
\end{center}

\setlength{\unitlength}{1in}
\begin{picture}(6,.1) 
\put(0,0) {\line(1,0){6.25}}         
\end{picture}

%--------------------------------------------------------------------
%\section{Differentiation}
%Sometimes rather than calculating values of a function, we need to approximate or interpolate them instead. Why might we do this?
%%
%\begin{itemize}
%\item It may be difficult or impossible to analytically evaluate the function
%\item We may have only a table of values at certain points and need to
%determine the values in between
%\item It may be much faster to compute values of an approximation
%function than of the original function, particularly if we have to
%calculate the function values over and over again
%\item A function may be defined implicitly
%\end{itemize}

\begin{tikzpicture}
\draw (-.25,0)--(1.25,0);
\draw[dotted] (1.25,0)--(2.75,0);
\draw (2.75,0)--(5.25,0);
\draw[dotted] (5.25,0)--(6.75,0);
\draw (6.75,0)--(8.25,0);
%\draw (4,0)--(5.25,0);
\draw (0,-.25)--(0,.25);
\node[below] at (0,-.25) {$x_0$}
\draw (1,-.25)--(1,.25);
%\draw (2,-.25)--(2,.25);
\draw (3,-.25)--(3,.25);
\draw (4,-.25)--(4,.25);
\draw (5,-.25)--(5,.25);
\draw (7,-.25)--(7,.25);
\draw (8,-.25)--(8,.25);
\end{tikzpicture}



%--------------------------------------------------------------------
%--------------------------------------------------------------------
%\bibliographystyle{plain}
%\bibliography{LinearSolns} 

\end{document}
