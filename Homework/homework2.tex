%--------------------------------------------------------------------
% NE 155 (intro to numerical simulation of radiation transport)
% Homework 2
% Spring 2014

% Exam Template from UMTYMP and Math Department courses
%
% Using Philip Hirschhorn's exam.cls: http://www-math.mit.edu/~psh/#ExamCls
%
% run pdflatex on a finished exam at least three times to do the grading table on front page.

%
%%%%%%%%%%%%%%%%%%%%%%%%%%%%%%%%%%%%%%%%%%%%%%%%%%%%%%%%%%%%%%%%%%%%%%%%%%%%%%%%%%%%%%%%%%%%%%

% These lines can probably stay unchanged, although you can remove the last
% two packages if you're not making pictures with tikz.
\documentclass[12pt, answers]{exam}
\RequirePackage{amssymb, amsfonts, amsmath, latexsym, verbatim, xspace, setspace}
\RequirePackage{tikz}
\usetikzlibrary {plotmarks}

% By default LaTeX uses large margins.  This doesn't work well on exams; problems
% end up in the "middle" of the page, reducing the amount of space for students
% to work on them.
\usepackage[margin=1in]{geometry}
\usepackage{enumerate}


% Here's where you edit the Class, Exam, Date, etc.
\newcommand{\class}{NE 155}
\newcommand{\term}{Spring 2014}
\newcommand{\assignment}{HW 2}
\newcommand{\duedate}{2/19/14}
%\newcommand{\timelimit}{50 Minutes}

\newcommand{\nth}{n\ensuremath{^{\text{th}}} }
\newcommand{\ve}[1]{\ensuremath{\mathbf{#1}}}
\newcommand{\Macro}{\ensuremath{\Sigma}}
\newcommand{\vOmega}{\ensuremath{\hat{\Omega}}}

% For an exam, single spacing is most appropriate
\singlespacing
% \onehalfspacing
% \doublespacing

% For an exam, we generally want to turn off paragraph indentation
\parindent 0ex

\begin{document} 

% These commands set up the running header on the top of the exam pages
\pagestyle{head}
\firstpageheader{}{}{}
\runningheader{\class}{\assignment\ - Page \thepage\ of \numpages}{Due \duedate}
\runningheadrule

\begin{flushright}
\begin{tabular}{p{5in} r l}
NE 155 & Spring 2014 \\
Homework \#2 & Due February 19, 2014
\end{tabular}
\end{flushright}
\rule[1ex]{\textwidth}{.1pt}

%%%%%%%%%%%%%%%%%%%%%%%%%%%%%%%%%%%%%%%%%%%%%%%%%%%%%%%%%%%%%%%%%%%%%%%%%%%%%%%%%%%%%
%
% See http://www-math.mit.edu/~psh/#ExamCls for full documentation, but the questions
% below give an idea of how to write questions [with parts] and have the points
% tracked automatically on the cover page.
%
%
%%%%%%%%%%%%%%%%%%%%%%%%%%%%%%%%%%%%%%%%%%%%%%%%%%%%%%%%%%%%%%%%%%%%%%%%%%%%%%%%%%%%%

Show your work so that partial credit may be awarded.

\begin{questions}

\addpoints
\question[10] 
Determine which of the following matrices are non-singular and compute the inverse of these matrices:
% a values
\newcommand{\aaa}{4}
\newcommand{\aab}{2}
\newcommand{\aac}{6}
\newcommand{\aba}{3}
\newcommand{\abb}{0}
\newcommand{\abc}{7}
\newcommand{\aca}{-2}
\newcommand{\acb}{-1}
\newcommand{\acc}{-1}
% b values
\newcommand{\baa}{2}
\newcommand{\bab}{0}
\newcommand{\bac}{0}
\newcommand{\bba}{0}
\newcommand{\bbb}{-3}
\newcommand{\bbc}{0}
\newcommand{\bca}{0}
\newcommand{\bcb}{0}
\newcommand{\bcc}{1}
\begin{equation}
\text{a.} \begin{pmatrix}
   \aaa & \aab & \aac \\
   \aba & \abb & \abc \\
   \aca & \acb & \acc \\
\end{pmatrix} \qquad
%
\text{b.} \begin{pmatrix}
   \baa & \bab & \bac \\
   \bba & \bbb & \bbc \\
   \bca & \bcb & \bcc \\
\end{pmatrix} \qquad
%
\text{c.} \begin{pmatrix}
  1  & 1 & -1 & -1 \\
  1  & 2 & -4 & -2 \\
  2  & 1 &  1 & 5 \\
  -1 & 0 & -2 & -4
\end{pmatrix} \nonumber
\end{equation}

\begin{solution}
A way to tell if $\ve{A}$ is singular that we discussed in class is that the determinant of $\ve{A}=0$.

\begin{parts}
\part Since \ve{a} has a zero on the diagonal, we know that it's determinant is zero. Thus, \ve{a} is singular.

\part Computing the determinant of \ve{b} is easy since it's diagonal:
\begin{align}
\det(\ve{b}) &= \baa ( \bbb \cdot \bcc ) - \bbb ( \baa \cdot \bcc ) + \bcc ( \baa \cdot \bbb ) \nonumber \\
\det(\ve{b}) &= -6 + 6 -6 = -6 \rightarrow \boxed{\text{non-singular}}\nonumber
\end{align}
%
Since \ve{b} is non-singular, we only need to compute it's inverse. Recall that for the inverse of a diagonal matrix $d^{-1}_{ii} = 1/d_{ii}$. Thus,
\[\ve{b}^{-1} = \begin{pmatrix}
  \frac{1}{2} &  0 & 0 \\
  0 & -\frac{1}{3} & 0 \\
  0 &  0 & 1
\end{pmatrix}\]

\part 

\end{parts}


\end{solution}


% ---------------------------------------------
\vspace*{3em}
\addpoints
\question[10] Determine the eigenvalues and associated eigenvectors of the following matrices:
%
\begin{equation}
\text{a.} \begin{pmatrix}
   2  & -1 \\
   -1 &  2  
\end{pmatrix} \qquad
%
\text{b.} \begin{pmatrix}
  2 &  1 & 0 \\
  1 &  2 & 0 \\
  0 &  0 & 3
\end{pmatrix} \qquad 
\text{c.} \begin{pmatrix}
  3  &  2 & -1 \\
  1  & -2 &  3 \\
  2  &  0 &  4
\end{pmatrix} \nonumber
\end{equation}

\begin{solution}
To find eigenvalues, we need to compute $\det(\ve{A} - \lambda \ve{I})=0$ for each system.



\end{solution}


%------------------------------------
\vspace*{3em}
\addpoints
\question[10] We have three systems of linear equations that are similar but different. Of them, \underline{one} has an exact solution, \underline{one} has infinitely many solutions, and \underline{one} has no solution. 
%
\begin{enumerate}[a.]
\item Determine which system is which.
\item Discuss the approach(es) you would use to solve these systems by hand.
\item Find the solutions (as applicable). You may use MATLAB or Python to solve these systems. 
\end{enumerate}
%
\begin{enumerate}
\item 
\begin{align}
4 x_1 -   x_2 + 2 x_3 + 3 x_4 &= 20 \nonumber \\
      - 2 x_2 + 7 x_3 - 4 x_4 &= -7 \nonumber \\
                6 x_3 + 5 x_4 &= 4  \nonumber \\
                        3 x_4 &= 6  \nonumber
\end{align}

\item 
\begin{align}
4 x_1 -   x_2 + 2 x_3 + 3 x_4 &= 20 \nonumber \\
        0 x_2 + 7 x_3 - 4 x_4 &= -7 \nonumber \\
                6 x_3 + 5 x_4 &= 4  \nonumber \\
                        3 x_4 &= 6  \nonumber
\end{align}

\item 
\begin{align}
4 x_1 -   x_2 + 2 x_3 + 3 x_4 &= 20 \nonumber \\
        0 x_2 + 7 x_3 + 0 x_4 &= -7 \nonumber \\
                6 x_3 + 5 x_4 &= 4  \nonumber \\
                        3 x_4 &= 6  \nonumber
\end{align}
\end{enumerate}


% ---------------------------------------------
\vspace*{3em}
\addpoints
\question[10] Find the parabola
\[ y = a + bx + cx^2\]
that passes through the points $(1,1)$, $(2,-1)$, and $(3,1)$. 

Use Gaussian elimination and backward substitution as your solution technique.


% ---------------------------------------------
\vspace*{3em}
\addpoints
\question[10] Determine the spectral radii of the following matrices:
%
\begin{equation}
\text{a.} \begin{pmatrix}
   0   & 1/2 \\
   1/2 &  0  
\end{pmatrix} \qquad
%
\text{b.} \begin{pmatrix}
  -1 &  2 & 0 \\
   0 &  3 & 4 \\
   0 &  0 & 7
\end{pmatrix} \nonumber
\end{equation}
%
Discuss the significance of the spectral radius for the iterative solution of $\ve{A}\vec{x} = \vec{b}$.


% ---------------------------------------------
\vspace*{3em}
\addpoints
\question[10] Find the LU Decomposition of $\ve{A}$ using Gaussian elimination and use it to solve $\ve{A}\vec{x} = \vec{b}$.
%
\begin{equation}
\ve{A} = \begin{pmatrix}
  10 & -7 & 0 \\
  -3 &  2 & 6 \\
   5 & -1 & 5
\end{pmatrix} \; \qquad
%
\vec{b} = \begin{pmatrix} 7 \\ 4 \\ 6 \end{pmatrix} \nonumber
\end{equation}


\end{questions}

\end{document}
