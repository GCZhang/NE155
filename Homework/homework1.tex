%--------------------------------------------------------------------
% NE 155 (intro to numerical simulation of radiation transport)
% Homework 1
% Spring 2014

% Exam Template from UMTYMP and Math Department courses
%
% Using Philip Hirschhorn's exam.cls: http://www-math.mit.edu/~psh/#ExamCls
%
% run pdflatex on a finished exam at least three times to do the grading table on front page.
%
%%%%%%%%%%%%%%%%%%%%%%%%%%%%%%%%%%%%%%%%%%%%%%%%%%%%%%%%%%%%%%%%%%%%%%%%%%%%%%%%%%%%%%%%%%%%%%

% These lines can probably stay unchanged, although you can remove the last
% two packages if you're not making pictures with tikz.
\documentclass[12pt, answers]{exam}
\RequirePackage{amssymb, amsfonts, amsmath, latexsym, verbatim, xspace, setspace}
\RequirePackage{tikz, pgflibraryplotmarks}

% By default LaTeX uses large margins.  This doesn't work well on exams; problems
% end up in the "middle" of the page, reducing the amount of space for students
% to work on them.
\usepackage[margin=1in]{geometry}
\usepackage{enumerate}


% Here's where you edit the Class, Exam, Date, etc.
\newcommand{\class}{NE 155}
\newcommand{\term}{Spring 2014}
\newcommand{\assignment}{HW 1}
\newcommand{\duedate}{2/5/14}
%\newcommand{\timelimit}{50 Minutes}

\newcommand{\nth}{n\ensuremath{^{\text{th}}} }
\newcommand{\ve}[1]{\ensuremath{\mathbf{#1}}}
\newcommand{\Macro}{\ensuremath{\Sigma}}
\newcommand{\vOmega}{\ensuremath{\hat{\Omega}}}

% For an exam, single spacing is most appropriate
\singlespacing
% \onehalfspacing
% \doublespacing

% For an exam, we generally want to turn off paragraph indentation
\parindent 0ex

\begin{document} 

% These commands set up the running header on the top of the exam pages
\pagestyle{head}
\firstpageheader{}{}{}
\runningheader{\class}{\assignment\ - Page \thepage\ of \numpages}{Due \duedate}
\runningheadrule

\begin{flushright}
\begin{tabular}{p{5in} r l}
NE 155 & Spring 2014 \\
Homework \#1 & Due February 5, 2014
\end{tabular}
\end{flushright}
\rule[1ex]{\textwidth}{.1pt}

%%%%%%%%%%%%%%%%%%%%%%%%%%%%%%%%%%%%%%%%%%%%%%%%%%%%%%%%%%%%%%%%%%%%%%%%%%%%%%%%%%%%%
%
% See http://www-math.mit.edu/~psh/#ExamCls for full documentation, but the questions
% below give an idea of how to write questions [with parts] and have the points
% tracked automatically on the cover page.
%
%
%%%%%%%%%%%%%%%%%%%%%%%%%%%%%%%%%%%%%%%%%%%%%%%%%%%%%%%%%%%%%%%%%%%%%%%%%%%%%%%%%%%%%

\begin{questions}

\addpoints
\question 
\begin{parts}
\part[5] Why are you taking this class? 
\part[5] Is there a specific, non-grade related outcome you would like?
\end{parts}


% ---------------------------------------------
\vspace*{2em}
\addpoints
\question 
\begin{parts}
\part[5] In your opinion, what was the most interesting development in the history of computing? 
\part[5] Why?
\end{parts}


% ---------------------------------------------
\vspace*{2em}
\addpoints
\question[10] Describe a challenge you can imagine having when writing a program to use \textbf{shared} memory.

\begin{solution}
Some common challenges when using \textbf{shared} memory include:
\begin{itemize}
\item race conditions: two processes are trying to modify/access the same data, e.g.:
\begin{itemize}
\item Two processes are trying to access the same data at the same time 
\item Two processes are trying to write to the same location in memory at the same time 
\item One process is trying to write to a location that another process is trying to read
\end{itemize}
\item CPU-to-memory connection becomes a bottleneck.
\item Cache coherence: each processor has it's own cache even though the memory is shared. There can be issues with mis-matches between the cache and the local memory. 
\end{itemize}
\end{solution}

\newpage
%------------------------------------
\vspace*{2em}
\addpoints
\question[10] Describe a challenge you can imagine having when writing a program to use \textbf{distributed} memory.

\begin{solution}
Some common challenges when using \textbf{distributed} memory include:
\begin{itemize}
\item Each process has to know where the data it needs resides
\item Accessing data in memory on other nodes can be slow
\item Deciding how to most efficiently \underline{distribute} data is required
\item Deciding how to most efficiently \underline{access} data is required
\item The best way to distribute/access the data will probably change when the network topology (how the memory spaces are connected) changes
\end{itemize}
\end{solution}


%------------------------------------
\vspace*{2em}
\addpoints
\question[20] Briefly describe what each term in the Transport Equation [ eqn.~\eqref{eq:TE}] physically represents.

\begin{align}
[\underbrace{\hat{\Omega} \cdot \nabla\psi(\vec{r}, \hat{\Omega}, E)}_A &+
%
 \underbrace{\Macro(\vec{r}, E)\psi(\vec{r}, \hat{\Omega}, E)}_B   = 
%
 \underbrace{\int_0^{\infty} dE' \int_{4\pi} d\hat{\Omega'} \:\Macro_{s}(\vec{r}, E' \to E, \hat{\Omega'} \cdot \hat{\Omega}) \psi(\vec{r}, \hat{\Omega'}, E')}_C \nonumber \\
%
&+\underbrace{\frac{ \chi(E)}{k} \int_0^{\infty} dE' \:\nu \Macro_{f}(\vec{r}, E') \int_{4\pi} d\hat{\Omega'} \:\psi(\vec{r}, \hat{\Omega'}, E')}_D
\label{eq:TE}
\end{align}

\begin{solution}
\begin{enumerate}[A.]
\item Streaming loss term: neutrons move to locations and/or directions of travel other than the one(s) we're interested in
\item Total interaction loss term: neutrons experience an interaction and are absorbed or scatter into an energy or angle other than the one(s) we're interested in
\item Scattering source term: neutrons scatter into the energy and direction of interest from any other energy and direction
\item Fission source term: neutrons born at the location, energy, and angle of interest from fission
\end{enumerate}

\end{solution}

%------------------------------------
\vspace*{2em}
\addpoints
\question[10] List a major assumption needed to get from the Transport Equation to the Diffusion Equation.

\begin{solution}
There are several assumptions made in the derivation; listed in the order we used them in class.
%
\begin{enumerate}
\item Angular dependence of the flux is unimportant (used when we integrated the TE over angle)

\item Angular flux is weakly, in fact linearly, dependent on angle rather than independent of angle (used after obtaining the first moment of the TE; we expanded angular flux in angle and retained up through the term linearly dependent upon angle)

\item $[$less major$]$ The anisotropic contribution to energy transfer in a scattering collision can be neglected (used to simplify the 1st moment of the scattering cross section, $\Macro_{s1}$)

\item $[$less major$]$ The external source is isotropic (used to say the 1st moment of the external source is 0)

\end{enumerate}
%
\end{solution}


% ---------------------------------------------
\vspace*{2em}
\addpoints
\question[10] List four locations where the Diffusion Equation is not valid because the underlying assumptions do not hold.

\begin{solution}
The diffusion equations is not valid near
%
\begin{enumerate}
\item a void, 
\item boundary, 
\item source, 
\item or strong absorber.
\end{enumerate} 
%
\end{solution}

\end{questions}

\end{document}
