%--------------------------------------------------------------------
% NE 155 (intro to numerical simulation of radiation transport)
% Homework 1
% Spring 2014

% Exam Template from UMTYMP and Math Department courses
%
% Using Philip Hirschhorn's exam.cls: http://www-math.mit.edu/~psh/#ExamCls
%
% run pdflatex on a finished exam at least three times to do the grading table on front page.
%
%%%%%%%%%%%%%%%%%%%%%%%%%%%%%%%%%%%%%%%%%%%%%%%%%%%%%%%%%%%%%%%%%%%%%%%%%%%%%%%%%%%%%%%%%%%%%%

% These lines can probably stay unchanged, although you can remove the last
% two packages if you're not making pictures with tikz.
\documentclass[12pt]{exam}
\RequirePackage{amssymb, amsfonts, amsmath, latexsym, verbatim, xspace, setspace}
\RequirePackage{tikz, pgflibraryplotmarks}

% By default LaTeX uses large margins.  This doesn't work well on exams; problems
% end up in the "middle" of the page, reducing the amount of space for students
% to work on them.
\usepackage[margin=1in]{geometry}


% Here's where you edit the Class, Exam, Date, etc.
\newcommand{\class}{NE 155}
\newcommand{\term}{Spring 2014}
\newcommand{\assignment}{HW 1}
\newcommand{\duedate}{2/5/14}
%\newcommand{\timelimit}{50 Minutes}

\newcommand{\nth}{n\ensuremath{^{\text{th}}} }
\newcommand{\ve}[1]{\ensuremath{\mathbf{#1}}}
\newcommand{\Macro}{\ensuremath{\Sigma}}
\newcommand{\vOmega}{\ensuremath{\hat{\Omega}}}

% For an exam, single spacing is most appropriate
\singlespacing
% \onehalfspacing
% \doublespacing

% For an exam, we generally want to turn off paragraph indentation
\parindent 0ex

\begin{document} 

% These commands set up the running header on the top of the exam pages
\pagestyle{head}
\firstpageheader{}{}{}
\runningheader{\class}{\assignment\ - Page \thepage\ of \numpages}{Due \duedate}
\runningheadrule

%\begin{flushright}
%\begin{tabular}{p{2.8in} r l}
%\textbf{\class} & \textbf{Name (Print):} & \makebox[2in]{\hrulefill}\\
%\textbf{\term} &&\\
%\textbf{\assignment} &&\\
%\textbf{Due \duedate} &&\\
%%\textbf{Time Limit: \timelimit} & Teaching Assistant & \makebox[2in]{\hrulefill}
%\end{tabular}\\
%\end{flushright}
%\rule[1ex]{\textwidth}{.1pt}

\begin{flushright}
\begin{tabular}{p{5in} r l}
NE 155 & Spring 2014 \\
Homework \#1 & Due February 5, 2014
\end{tabular}
\end{flushright}
\rule[1ex]{\textwidth}{.1pt}

%%%%%%%%%%%%%%%%%%%%%%%%%%%%%%%%%%%%%%%%%%%%%%%%%%%%%%%%%%%%%%%%%%%%%%%%%%%%%%%%%%%%%
%
% See http://www-math.mit.edu/~psh/#ExamCls for full documentation, but the questions
% below give an idea of how to write questions [with parts] and have the points
% tracked automatically on the cover page.
%
%
%%%%%%%%%%%%%%%%%%%%%%%%%%%%%%%%%%%%%%%%%%%%%%%%%%%%%%%%%%%%%%%%%%%%%%%%%%%%%%%%%%%%%

\begin{questions}

% Basic question
\addpoints
\question[10] Differentiate $f(x)=x^2$ with respect to $x$.

% Question with parts
\newpage
\addpoints
\question Consider the function $f(x)=x^2$.
\begin{parts}
\part[5] Find $f'(x)$ using the limit definition of derivative.
\vfill
\part[5] Find the line tangent to the graph of $y=f(x)$ at the point where $x=2$.
\vfill
\end{parts}

% If you want the total number of points for a question displayed at the top,
% as well as the number of points for each part, then you must turn off the point-counter
% or they will be double counted.
\newpage
\addpoints
\question[10] Consider the function $f(x)=x^3$.
\noaddpoints % If you remove this line, the grading table will show 20 points for this problem.
\begin{parts}
\part[5] Find $f'(x)$ using the limit definition of derivative.
\vspace{4.5in}
\part[5] Find the line tangent to the graph of $y=f(x)$ at the point where $x=2$.
\end{parts}



\end{questions}
\end{document}
