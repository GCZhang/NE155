%--------------------------------------------------------------------
% NE 155 (intro to numerical simulation of radiation transport)
% Homework 1
% Spring 2014

% Exam Template from UMTYMP and Math Department courses
%
% Using Philip Hirschhorn's exam.cls: http://www-math.mit.edu/~psh/#ExamCls
%
% run pdflatex on a finished exam at least three times to do the grading table on front page.
%
%%%%%%%%%%%%%%%%%%%%%%%%%%%%%%%%%%%%%%%%%%%%%%%%%%%%%%%%%%%%%%%%%%%%%%%%%%%%%%%%%%%%%%%%%%%%%%

% These lines can probably stay unchanged, although you can remove the last
% two packages if you're not making pictures with tikz.
\documentclass[12pt]{exam}
\RequirePackage{amssymb, amsfonts, amsmath, latexsym, verbatim, xspace, setspace}
\RequirePackage{tikz, pgflibraryplotmarks}

% By default LaTeX uses large margins.  This doesn't work well on exams; problems
% end up in the "middle" of the page, reducing the amount of space for students
% to work on them.
\usepackage[margin=1in]{geometry}


% Here's where you edit the Class, Exam, Date, etc.
\newcommand{\class}{NE 155}
\newcommand{\term}{Spring 2014}
\newcommand{\assignment}{HW 1}
\newcommand{\duedate}{2/5/14}
%\newcommand{\timelimit}{50 Minutes}

\newcommand{\nth}{n\ensuremath{^{\text{th}}} }
\newcommand{\ve}[1]{\ensuremath{\mathbf{#1}}}
\newcommand{\Macro}{\ensuremath{\Sigma}}
\newcommand{\vOmega}{\ensuremath{\hat{\Omega}}}

% For an exam, single spacing is most appropriate
\singlespacing
% \onehalfspacing
% \doublespacing

% For an exam, we generally want to turn off paragraph indentation
\parindent 0ex

\begin{document} 

% These commands set up the running header on the top of the exam pages
\pagestyle{head}
\firstpageheader{}{}{}
\runningheader{\class}{\assignment\ - Page \thepage\ of \numpages}{Due \duedate}
\runningheadrule

\begin{flushright}
\begin{tabular}{p{5in} r l}
NE 155 & Spring 2014 \\
Homework \#1 & Due February 5, 2014
\end{tabular}
\end{flushright}
\rule[1ex]{\textwidth}{.1pt}

%%%%%%%%%%%%%%%%%%%%%%%%%%%%%%%%%%%%%%%%%%%%%%%%%%%%%%%%%%%%%%%%%%%%%%%%%%%%%%%%%%%%%
%
% See http://www-math.mit.edu/~psh/#ExamCls for full documentation, but the questions
% below give an idea of how to write questions [with parts] and have the points
% tracked automatically on the cover page.
%
%
%%%%%%%%%%%%%%%%%%%%%%%%%%%%%%%%%%%%%%%%%%%%%%%%%%%%%%%%%%%%%%%%%%%%%%%%%%%%%%%%%%%%%

\begin{questions}

\addpoints
\question 
\begin{parts}
\part[5] Why are you taking this class? 
\part[5] Is there a specific, non-grade related outcome you would like?
\end{parts}

\vspace*{2em}
\addpoints
\question 
\begin{parts}
\part[5] In your opinion, what was the most interesting development in the history of computing? 
\part[5] Why?
\end{parts}

\vspace*{2em}
\addpoints
\question[10] Describe a challenge you can imagine having when writing a program to use \textbf{shared} memory.

\vspace*{2em}
\addpoints
\question[10] Describe a challenge you can imagine having when writing a program to use \textbf{distributed} memory.

%------------------------------------
\vspace*{2em}
\addpoints
\question[20] Briefly describe what each term in the Transport Equation [ eqn.~\eqref{eq:TE}] physically represents.

\begin{align}
[\underbrace{\hat{\Omega} \cdot \nabla\psi(\vec{r}, \hat{\Omega}, E)}_A &+
%
 \underbrace{\Macro(\vec{r}, E)\psi(\vec{r}, \hat{\Omega}, E)}_B   = 
%
 \underbrace{\int_0^{\infty} dE' \int_{4\pi} d\hat{\Omega'} \:\Macro_{s}(\vec{r}, E' \to E, \hat{\Omega'} \cdot \hat{\Omega}) \psi(\vec{r}, \hat{\Omega'}, E')}_C \nonumber \\
%
&+\underbrace{\frac{ \chi(E)}{k} \int_0^{\infty} dE' \:\nu \Macro_{f}(\vec{r}, E') \int_{4\pi} d\hat{\Omega'} \:\psi(\vec{r}, \hat{\Omega'}, E')}_D
\label{eq:TE}
\end{align}

%------------------------------------
\vspace*{2em}
\addpoints
\question[10] List a major assumption needed to get from the Transport Equation to the Diffusion Equation.

\begin{solution}

\end{solution}

\vspace*{2em}
\addpoints
\question[10] List four locations where the Diffusion Equation is not valid because the underlying assumptions do not hold.

\end{questions}

\end{document}
