%--------------------------------------------------------------------
% NE 155 (intro to numerical simulation of radiation transport)
% Homework 3
% Spring 2014

% Exam Template from UMTYMP and Math Department courses
%
% Using Philip Hirschhorn's exam.cls: http://www-math.mit.edu/~psh/#ExamCls
%
% run pdflatex on a finished exam at least three times to do the grading table on front page.
%
%%%%%%%%%%%%%%%%%%%%%%%%%%%%%%%%%%%%%%%%%%%%%%%%%%%%%%%%%%%%%%%%%%%%%%%%%%%%%%%%%%%%%%%%%%%%%%

% These lines can probably stay unchanged, although you can remove the last
% two packages if you're not making pictures with tikz.
\documentclass[12pt, answers]{exam}
\RequirePackage{amssymb, amsfonts, amsmath, latexsym, verbatim, xspace, setspace}
\RequirePackage{tikz, pgflibraryplotmarks}

% By default LaTeX uses large margins.  This doesn't work well on exams; problems
% end up in the "middle" of the page, reducing the amount of space for students
% to work on them.
\usepackage[margin=1in]{geometry}
\usepackage{enumerate}


% Here's where you edit the Class, Exam, Date, etc.
\newcommand{\class}{NE 155}
\newcommand{\term}{Spring 2014}
\newcommand{\assignment}{HW 3}
\newcommand{\duedate}{2/28/14}
%\newcommand{\timelimit}{50 Minutes}

\newcommand{\nth}{n\ensuremath{^{\text{th}}} }
\newcommand{\ve}[1]{\ensuremath{\mathbf{#1}}}
\newcommand{\Macro}{\ensuremath{\Sigma}}
\newcommand{\vOmega}{\ensuremath{\hat{\Omega}}}

% For an exam, single spacing is most appropriate
\singlespacing
% \onehalfspacing
% \doublespacing

% For an exam, we generally want to turn off paragraph indentation
\parindent 0ex

\begin{document} 

% These commands set up the running header on the top of the exam pages
\pagestyle{head}
\firstpageheader{}{}{}
\runningheader{\class}{\assignment\ - Page \thepage\ of \numpages}{Due \duedate}
\runningheadrule

\begin{flushright}
\begin{tabular}{p{5in} r l}
NE 155 & Spring 2014 \\
Homework \#3 & Due February 28, 2014
\end{tabular}
\end{flushright}
\rule[1ex]{\textwidth}{.1pt}

%%%%%%%%%%%%%%%%%%%%%%%%%%%%%%%%%%%%%%%%%%%%%%%%%%%%%%%%%%%%%%%%%%%%%%%%%%%%%%%%%%%%%
%
% See http://www-math.mit.edu/~psh/#ExamCls for full documentation, but the questions
% below give an idea of how to write questions [with parts] and have the points
% tracked automatically on the cover page.
%
%
%%%%%%%%%%%%%%%%%%%%%%%%%%%%%%%%%%%%%%%%%%%%%%%%%%%%%%%%%%%%%%%%%%%%%%%%%%%%%%%%%%%%%

\begin{questions}

\addpoints
\question[10]
For $n=100$ we will use this tridiagonal system of equations 
%
\begin{align}
&2x_0 - x_1 = 1 \nonumber \\
%
-x_{j-1} + &2x_j - x_{j+1} = j+1\:, \qquad j = 1, \dots, n-2 \nonumber \\
%
-x_{n-2} + &2x_n = n-1 \nonumber
\end{align}
%
in a few different ways. You may use Python or MATLAB; if you would like to use another package or code language please consult with me. Note: NumPy and SciPy are libraries that can be imported into Python and would be useful for this assignment.  
%
%The directions will be given as both Python and MATLAB commands. In Python you can use \textit{diags} from SciPy's \textit{sparse} package to create the matrix and NumPy's \textit{arange} to create $\vec{b}$; in MATLAB you can use \textit{diag} or \textit{spdiags}. 

\noaddpoints
\begin{parts}
\part[2.5] Use built in Python or MATLAB commands to construct \ve{A} and $\vec{b}$.
\part[2.5] What is the condition number of \ve{A}? 
\part[2.5] Solve this problem by explicitly inverting \ve{A} and multiplying $\vec{b}$.
\part[2.5] Use scipy.linalg.solve (Python) or the backslash operator (MATLAB) to solve the system.
\end{parts}

Plot both solutions on the same plot. Include axis labels, a title, and a legend. 

% -----------------------------------------------------------
\vspace*{3em}
\addpoints
\question[10]
\label{Q:iter}
Find the first two iterations (report the vectors $x_1$ and $x_2$) of the
%
\begin{parts}
\part Jacobi method
\part Gauss Seidel method
\part SOR (with $\omega = 1.1$) method
\end{parts}
%
for the following two systems of equations. Use $x_0 = 0$.
%
% a values
\newcommand{\aaa}{3}
\newcommand{\aab}{-1}
\newcommand{\aac}{1}
\newcommand{\aba}{3}
\newcommand{\abb}{6}
\newcommand{\abc}{2}
\newcommand{\aca}{3}
\newcommand{\acb}{3}
\newcommand{\acc}{7}
% a's solution
\newcommand{\ba}{1}
\newcommand{\bb}{0}
\newcommand{\bc}{4}
% c values
\newcommand{\caa}{2}
\newcommand{\cab}{-2}
\newcommand{\cac}{1}
\newcommand{\cad}{1}
\newcommand{\cba}{0}
\newcommand{\cbb}{-3}
\newcommand{\cbc}{0.5}
\newcommand{\cbd}{1}
\newcommand{\cca}{0}
\newcommand{\ccb}{0}
\newcommand{\ccc}{5}
\newcommand{\ccd}{-1}
\newcommand{\cda}{0}
\newcommand{\cdb}{0}
\newcommand{\cdc}{0}
\newcommand{\cdd}{2}
% c's solution
\newcommand{\ca}{0.8}
\newcommand{\cb}{-6.6}
\newcommand{\cc}{4.5}
\newcommand{\cd}{2}
%
\begin{alignat}{2}
&\aaa x_1 + \aab x_2 + \aac x_3 = \ba \qquad\qquad 
&\caa x_1 + \cab x_2 + \cac x_3 + \cad x_4 = \ca \nonumber \\
%
&\aba x_1 + \abb x_2 + \abc x_3 = \bb \qquad\qquad 
&\cba x_1 + \cbb x_2 + \cbc x_3 + \cbd x_4 = \cb \nonumber \\
%
&\aca x_1 + \acb x_2 + \acc x_3 = \bc \qquad \qquad
&\cca x_1 + \ccb x_2 + \ccc x_3 + \ccd x_4 = \cc \nonumber \\
%
& \qquad \qquad
&\cda x_1 + \cdb x_2 + \cdc x_3 + \cdd x_4 = \cd \nonumber
%
\end{alignat}


% -----------------------------------------------------------
\vspace*{3em}
\addpoints
\question[50]

We will use the following system of $n$ equations:
%
\newcommand{\dd}{4}
\newcommand{\dl}{-1}
\newcommand{\du}{-1}
\newcommand{\db}{100}
\begin{equation}
\ve{A}\vec{x} \equiv 
\begin{pmatrix}
      \dd    & \du    & 0      & \cdots & 0 \\
      \dl    & \dd    & \du    & \ddots & \vdots \\
      0      & \dl    & \dd    & \ddots & 0 \\     
      \vdots & \ddots & \ddots & \ddots & \du \\
      0      & \cdots & 0      & \dl    & \dd \\
    \end{pmatrix} 
\begin{pmatrix} x_0 \\ x_1 \\ x_2 \\ \vdots \\ x_{n-1} \end{pmatrix} 
=
\begin{pmatrix} \db \\ \db \\ \db \\ \vdots \\ \db \end{pmatrix} \equiv \vec{b} \nonumber
\end{equation}

\begin{parts}
\part[30] Write a program(s) to implement the Jacobi, Gauss Seidel, and SOR (with $\omega = 1.1$) methods for a matrix with $n$ unknowns. Turn in your source code electronically; include instructions for how to run it, input files, etc.\ if necessary.

\vspace*{1em}
Solve this system of equations with each method using the program(s) that you write. Use $x_0 = 0$ and  $n=5$. Print the solution vector from each method converged to an \textbf{absolute} tolerance of $10^{-6}$. Indicate the number of iterations required to meet this tolerance for each method. 

\vspace*{2em}
BONUS: submit your code by providing read/clone access to an online version control repository where your code is stored (e.g. github or bitbucket). If you don't know what that means and want to learn about it, come talk to me.

% --------
\vspace*{2em}
\part[10] How many iterations are required for each method to reach the stopping criterion:
%
\begin{equation}
\frac{||x^{(k+1)} - x^{(k)}||}{||x^{(k+1)}||} < \epsilon \nonumber
\end{equation}
%
for $\epsilon = 10^{-6}$ and $\epsilon = 10^{-8}$? Which method required the fewest iterations? Do you observe any trends in the convergence patterns?

% --------
\vspace*{2em}
\part[10] Perform an experiment to determine $\omega_{opt}$ for SOR. Explain your procedure and include the results. 

\end{parts}


\end{questions}

\end{document}
