\documentclass[12pt]{article}
%\textwidth=7in
%\textheight=9.5in
%\topmargin=-1in
%\headheight=0in
%\headsep=.5in
%\hoffset=-.85in

%\usepackage[cm]{fullpage}
\usepackage[top=0.75in, bottom=0.75in, left=1in, right=1in]{geometry}
\pagestyle{empty}
\usepackage{tabu}

\renewcommand{\thefootnote}{\fnsymbol{footnote}}
\begin{document}

\begin{center}
{\bf NE 155 - Introduction to Numerical Simulations in Radiation Transport \\ MWF 11:00 AM - 12:00 PM,  Room:  ???  
}
\end{center}

\setlength{\unitlength}{1in}
\begin{picture}(6,.1) 
\put(0,0) {\line(1,0){6.25}}         
\end{picture}

\renewcommand{\arraystretch}{2}

\vskip.25in
\noindent\textbf{Instructor:} Rachel Slaybaugh,  4173 Etcheverry Hall,\\ \hspace*{0.9 in}slaybaugh@berkeley.edu,  570-850-3385
\vskip.25in
\noindent\textbf{Office Hours:} 2:00-3:00 PM MF and by  appointment.

\vskip.25in
\noindent\textbf{GSI:} details
\vskip.25in
\noindent\textbf{GSI Office Hours:} details.

\vskip.25in
\noindent\textbf{Course Description:}
Computational methods used to analyze radiation transport described by various differential, integral, and integro�differential equations. Numerical methods include finite difference, finite elements, discrete ordinates, and Monte Carlo. Examples from neutron and photon transport; numerical solutions of neutron/photon diffusion and transport equations. Monte Carlo simulations of photon and neutron transport. An overview of optimization techniques for solving the resulting discrete equations on vector and parallel computer systems.

\vskip.25in
\noindent\textbf{Prerequisites:} Math 53 and 54 (Eng 7 recommended)

\vskip.25in
\noindent\textbf{Textbooks and References:}
\begin{itemize}
\item Course notes + handouts
\item Either \emph{Numerical Computing with MATLAB} by Cleve Moler \\(http://www.mathworks.com/moler/index\_ncm.html)
\item - OR - Choose a Python Ebook that fits your needs: http://www.leettips.org/2013/02/top-10-free-python-pdf-ebooks-download.html
\item E. E. Lewis, W. E. Miller Jr., ``Computational Methods of Neutron Transport," J. Wiley \& Sons, 1984.
\end{itemize}

\vskip.25in
\noindent\textbf{Grading:} 
\begin{itemize}
\item Homework 40\%
\item Midterms (2) 15\% + 15\% = 30\%
\item Final Project 30\% 
\item Late submissions: -20\% for each day it is late
\end{itemize}

\clearpage
\noindent\textbf{Computer Information:} 
\begin{itemize}
\item All students will get class computer lab accounts at Davis Etcheverry Computing Facility (DECF) (1171 and 1111 Etcheverry): http://www.decf.berkeley.edu/
\item License for MCNP6 (B00004-MNYCP-02) is obtained through RSICC:\\
http://rsicc.ornl.gov/Registration.aspx (apply for the EXE package only)
\end{itemize}

\vspace*{.15in}
\noindent \textbf{Course Outline:} 
\begin{enumerate}
\item Introduction
  \begin{enumerate}
  \item Overview of computational science/engineering 
  \item A brief history of computer technology 
  \item Types of differential and integral equations in radiation transport
    \begin{enumerate}
    \item Integro-�differential form of transport equation 
    \item Integral form of transport equation 
    \item Diffusion approximation to transport equation 
    \item Point �kinetics equation; depletion equation
    \end{enumerate}
  \item Review of numerical simulations: deterministic and probabilistic methods
  \end{enumerate}
\item Review of numerical analysis fundamentals
  \begin{enumerate}
  \item Basic linear algebra and matrix inversion
  \item Systems of linear algebraic equations
  \item Direct methods for solving linear systems; Gaussian elimination; pivoting strategies; techniques for special matrices
  \item Vector and matrix norms
  \item Eigenvalues and eigenvectors of a matrix; spectral radius of a matrix; convergence of vectors and matrices
  \item Iterative methods for solving linear systems: Jacobi, Gauss �Seidel and SOR Methods
  \item Interpolation and polynomial approximation review
  \item Numerical differentiation and integration review
  \end{enumerate}
\item Point �kinetics equation: numerical solution of initial value problem
  \begin{enumerate}
  \item Taylor method
  \item Runge�Kutta method
  \item Predictor-corrector methods
  \end{enumerate}
\item Neutron diffusion equation in 1-D: numerical solution of 2nd order ODEs
  \begin{enumerate}
  \item Formulation of the finite difference equations for the ``fixed-source" problem
  \item Direct solution by Gaussian elimination
  \item Iterative solutions by Jacobi, Gauss �Seidel, and SOR methods
  \item Formulation of the finite difference equation for the ``eigenvalue" (criticality) problem
  \item Power and inverse power iterative methods
  \end{enumerate}
\item Neutron transport equation in 1-D: numerical solution of integro-differential equations
  \begin{enumerate}
  \item Spatial discretization in slab geometry: diamond �difference, step difference, and step characteristic methods
  \item Angular discretization: discrete �ordinates ($S_N$) method, some $S_N$ Gauss-�Legendre quadrature sets
  \item Solution of fixed-�source problems with no scattering
  \item Iterative methods for solving discretized equations
  \item Source iteration for k-�eigenvalue problems
  \item Convergence of source iteration method
  \item Multidimensional discrete ordinates ($S_N$) methods (angular quadrants, ray effects, streaming effects)
  \end{enumerate}
\item Probabilistic numerical simulations: Monte Carlo method
  \begin{enumerate}
  \item Continuous and discrete probability distribution; probability density functions; cumulative probability distribution functions
  \item Random numbers; categories of random sampling
  \item Complex geometry description and ray tracing
  \item Analog Monte Carlo; non-analog Monte Carlo; importance sampling;  variance reduction methods; error estimates
  \item Monte Carlo simulation of neutron and photon transport
  \item Introduction to the MCNP or Serpent code
  \end{enumerate}   
\end{enumerate}

\vspace*{.15in}
\noindent\textbf{Academic Honesty}:  The University provides some basic guidance about academic integrity: http://sa.berkeley.edu/conduct/integrity. Lack of knowledge of the academic honesty policy is not a reasonable explanation for a violation. Questions related to course assignments and the academic honesty policy should be directed to the instructor.

\vskip.25in
\noindent\textbf{Extra Help}:  Do not hesitate to come to my office during office hours or by appointment to discuss a homework problem or any aspect of the course. 

\vskip.25in
\noindent\textbf{Attendance}: Students are expected to attend classes regularly. A student who incurs an excessive number of absences may be withdrawn from a class at the discretion of the professor.

%\vskip.25in
%\noindent\textbf{Schedule}:
%
%\begin{tabu}{| c | l | X | c |}\hline
%Lecture & Date & Topic & HW Due\\
%\hline
%1 & Jan 22 & Intro, organization, overview \& goals &   \\
%\hline
%2 & Jan 24 & Intro to computational science and engineering & \\                                                                  
%\hline
%3 & Jan 27 & Types of differential and integral equations in radiation transport & \\
%\hline
%4 & Jan 29 & Types of differential and integral equations in radiation transport cont'd & \\
%\hline
%5 & Jan 31 & Fundamentals of numerical methods - matrix algebra & \\
%\hline
%6 & Feb 3 & Vector and matrix norms; eigenvalues and eigenvectors of a matrix; spectral radius of a matrix & \\
%\hline
%7 & Feb 5 &  & 1 \\
%\hline 
%8 & Feb 7 & Approximation and interpolations & \\
%\hline 
%9 & Feb 10 & Numerical differentiation and integration & \\
%\hline 
%10 & Feb 12 & Legendre polynomials and Gauss-�Legendre quadratures & \\
%\hline 
%11 & Feb 14 & Direct solution by Gaussian elimination. & \\
%\hline 
%& Feb 17 & President's Day - no class & \\
%\hline 
%12 & Feb 19 & Jacobi, Gauss �Seidel and SOR methods & \\
%\hline 
%13 & Feb 21 & Introduction to Matlab & \\
%\hline 
%14 & Feb 24 & Intro to the finite difference method for 1-D diffusion & \\
%\hline 
%15 & Feb 26 & Intro to the finite volume method for 1-D diffusion & \\
% \hline 
%16 & Mar 1 & Fixed-source problems using finite difference and finiie volume methods & \\
%\hline 
%17 & Mar 3 & Eigenvalue problems using finite difference and finiie volume methods & \\
%\hline 
%18 & Mar 5 & Finite difference method for 1-D diffusion & \\
%\hline 
%19 & Mar 7 & Finite element method for 1-D diffusion & \\
%\hline 
%20 & Mar 10 & Introduction to Monte Carlo method & \\
%\hline 
%\end{tabu}
%
%\begin{tabu}{| c | l | X | c |}\hline
%21 & Mar 12 & Introduction to the MCNP & \\
%\hline 
%22 & Mar 14 & Analog Monte Carlo; non-analog Monte Carlo; categories of random sampling; ariance reduction methods & \\
%\hline 
%23 & Mar 17 & Monte Carlo � random number generators; basic statistics definitions; error estimates. & \\
%\hline 
%24 & Mar 19 & Monte Carlo method for neutron and photon transport & \\
%\hline 
%25 & Mar 21 & Monte Carlo method for neutron and photon transport & \\
%\hline 
%& Mar 24-28 & Spring Recess - no class & \\
% \hline  
%26 & Mar 31 &  & \\
%\hline 
%27 & Apr 2 &  & \\
%\hline 
%28 & Apr 4 &  & \\
%\hline 
%29 & Apr 7 &  & \\
%\hline 
%30 & Apr 9 &  & \\
%\hline 
%31 & Apr 11 &  & \\
%\hline 
%32 & Apr 14 &  & \\
%\hline 
%33 & Apr 16 &  & \\
%\hline 
%34 & Apr 18 &  & \\
%\hline 
%35 & Apr 21 &  & \\
%\hline 
%36 & Apr 23 &  & \\
%\hline 
%37 & Apr 25 &  & \\
%\hline 
%38 & Apr 28 &  & \\
%\hline 
%39 & Apr 30 &  & \\
%\hline 
%40 & May 2 &  & \\
%\hline 
%41 & May 5 &  & \\
%\hline 
%42 & May 7 &  & \\
%\hline 
%43 & May 9 &  & \\
%\hline 
%\end{tabu}

\end{document}