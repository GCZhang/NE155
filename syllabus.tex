\documentclass[12pt]{article}
\usepackage[top=0.75in, bottom=0.75in, left=1in, right=1in]{geometry}
\pagestyle{empty}
\usepackage{tabu}
\usepackage{hyperref}
\usepackage{csvsimple}

\renewcommand{\thefootnote}{\fnsymbol{footnote}}
\begin{document}

\begin{center}
{\bf NE 155 - Introduction to Numerical Simulations in Radiation Transport \\ MWF 11:00 AM - 12:00 PM,  Room: 285 Cory Hall  
}
\end{center}

\setlength{\unitlength}{1in}
\begin{picture}(6,.1) 
\put(0,0) {\line(1,0){6.25}}         
\end{picture}

\renewcommand{\arraystretch}{2}

\vskip.25in
\noindent\textbf{Instructor:} Rachel Slaybaugh,  4173 Etcheverry Hall,\\ \hspace*{0.95 in}slaybaugh@berkeley.edu,  570-850-3385
\vskip.25in
\noindent\textbf{Office Hours:} 2:30-3:30 PM Fridays and by appointment.

%\vskip.25in
%\noindent\textbf{GSI:} TBD
%\vskip.25in
%\noindent\textbf{GSI Office Hours:} TBD

\vskip.25in
\noindent\textbf{Course Description:}
Computational methods used to analyze radiation transport described by various differential, integral, and integro�differential equations. Numerical methods include finite difference, finite volume, discrete ordinates, and Monte Carlo. Examples from neutron and photon transport; numerical solutions of neutron/photon diffusion and transport equations. %Monte Carlo simulations of photon and neutron transport. An overview of optimization techniques for solving the resulting discrete equations on vector and parallel computer systems.

\vskip.25in
\noindent\textbf{Prerequisites:} Math 53 and 54 (Eng 7 or basic programming skills strongly recommended)

\vskip.25in
\noindent\textbf{References:}
\begin{itemize}
\item Course notes + handouts: \href{https://github.com/rachelslaybaugh/NE155}{https://github.com/rachelslaybaugh/NE155}
%\item E. E. Lewis, W. E. Miller Jr., ``Computational Methods of Neutron Transport," J. Wiley \& Sons, 1984.
\item Helpful Resources \href{http://tinyurl.com/ne-technical-resources}{http://tinyurl.com/ne-technical-resources}
\item Free Python Ebooks that might fits your needs: \href{http://www.leettips.org/2013/02/top-10-free-python-pdf-ebooks-download.html}{http://www.leettips.org/2013/02/top-10-free-python-pdf-ebooks-download.html}
\item Jupyter (the awesome thing formerly known as iPython): \href{http://jupyter.org/}{http://jupyter.org/}
\end{itemize}

\vskip.25in
\noindent\textbf{Grading:} 
\begin{itemize}
\item Homework 40\%
\item Midterms (2) 15\% + 15\% = 30\%
\item Final Project 30\% 
\item Late submissions: -20\% for each day it is late with a maximum of -60\%\footnote{The point of this is to try to get you to always do the homework}
\end{itemize}

%\vskip.25in
%\noindent\textbf{bCourses Site:} \href{https://bcourses.berkeley.edu/courses/1292802}{https://bcourses.berkeley.edu/courses/1292802}
%
%\noindent\textbf{Course GitHub page:} \href{https://github.com/rachelslaybaugh/NE155}{https://github.com/rachelslaybaugh/NE155}

\vskip.25in
\noindent\textbf{the Hacker Within:} 
\begin{itemize}
\item Wednesdays, 4-6 pm, 190 Doe Library (BIDS Space)
\item Will teach skills useful for this course
\item Website: \href{http://thehackerwithin.github.io/berkeley/}{http://thehackerwithin.github.io/berkeley/}
\item GitHub: \href{https://github.com/thehackerwithin/berkeley}{https://github.com/thehackerwithin/berkeley}
\end{itemize}

\clearpage
\noindent\textbf{Computer Information:} 
\begin{itemize}
  \item All students will get class computer lab accounts at Davis Etcheverry Computing Facility (DECF) (1171 and 1111 Etcheverry): \href{http://www.decf.berkeley.edu/}{http://www.decf.berkeley.edu/}
  \item A package with Python and many useful support libraries (called Anaconda) can be downloaded from \href{http://continuum.io/downloads}{http://continuum.io/downloads}
  \item Software Carpentry has useful lessons: \href{http://software-carpentry.org/lessons.html}{http://software-carpentry.org/lessons.html}
  \item We may use the Serpent Monte Carlo code (\href{http://montecarlo.vtt.fi/}{http://montecarlo.vtt.fi/}) in this course
%\item License for MCNP6 (B00004-MNYCP-02) is obtained through RSICC:\\
%http://rsicc.ornl.gov/Registration.aspx (apply for the EXE package only)
\end{itemize}

\vspace*{.15in}
\noindent \textbf{Useful Campus Information:} 
\begin{itemize}
  \item Mental health resources: \href{http://www.uhs.berkeley.edu/students/counseling/cps.shtml}{http://www.uhs.berkeley.edu/students/counseling/cps.shtml}
  \item Sexual assault support on campus: \href{http://survivorsupport.berkeley.edu/}{http://survivorsupport.berkeley.edu/}
\end{itemize}

\vspace*{.15in}
\noindent \textbf{Course Outline:} 
\begin{enumerate}
\item Introduction
  \begin{enumerate}
  \item Overview of computational science/engineering 
  \item History of computing and parallelization
  \item Types of differential and integral equations in radiation transport
%    \begin{enumerate}
%    \item Integro-�differential form of transport equation 
%    \item Integral form of transport equation 
%    \item Diffusion approximation to transport equation 
%    \item Point �kinetics equation; depletion equation
%    \end{enumerate}
  \item Overview of numerical simulations: deterministic and probabilistic methods
  \end{enumerate}

\item Numerical analysis fundamentals
  \begin{enumerate}
  \item Vector and matrix properties
  \item Eigenvalues and eigenvectors of a matrix; spectral radius of a matrix; convergence of vectors and matrices
  \item Interpolation and polynomial approximation
  \item Numerical differentiation and integration
  \item Direct methods for solving linear systems; Gaussian elimination; pivoting strategies; techniques for special matrices
  \item Iterative methods for solving linear systems: Jacobi, Gauss �Seidel and SOR 
  \end{enumerate}

\item Neutron diffusion equation in 1-D: numerical solution of 2nd order ODEs
  \begin{enumerate}
  \item Derivation of the transport and diffusion equations
  \item Formulation of the finite difference and volume equations for the ``fixed-source" problem
  \item Direct solution by Gaussian elimination
  \item Iterative solutions by Jacobi, Gauss �Seidel, and SOR 
  \item Formulation of the finite difference and volume equations for the eigenvalue (criticality) problem
  \item Power and inverse power iterative methods
  \item Extension to 2-D
  \end{enumerate}
 
 \clearpage 
\item Point �kinetics equation: numerical solution of initial value problem
  \begin{enumerate}
  \item Taylor method
  \item Runge�Kutta method
  \item Predictor-corrector methods
  \end{enumerate}

\item Probabilistic numerical simulations: Monte Carlo method
%  \begin{enumerate}
%  \item Continuous and discrete probability distribution; probability density functions; cumulative probability distribution functions
%  \item Random numbers; categories of random sampling
%  \item Complex geometry description and ray tracing
%  \item Analog Monte Carlo; non-analog Monte Carlo; importance sampling;  variance reduction methods; error estimates
%  \item Monte Carlo simulation of neutron and photon transport
%  \item Introduction to the MCNP or Serpent code
%  \end{enumerate} 
    
\item Neutron transport equation in 1-D: numerical solution of integro-differential equations
%  \begin{enumerate}
%  \item Spatial discretization in slab geometry: diamond �difference, step difference, and step characteristic methods
%  \item Angular discretization: discrete �ordinates ($S_N$) method, some $S_N$ Gauss-�Legendre quadrature sets
%  \item Solution of fixed-�source problems with no scattering
%  \item Iterative methods for solving discretized equations
%  \item Source iteration for k-�eigenvalue problems
%  \item Convergence of source iteration method
%  \item Multidimensional discrete ordinates ($S_N$) methods (angular quadrants, ray effects, streaming effects)
%  \end{enumerate}
\end{enumerate}

\vspace*{.15in}
\noindent\textbf{Academic Honesty}:  Berkeley's honor code is

\begin{quote}
As a member of the UC Berkeley community, I act with honesty, integrity, and respect for others.
\end{quote}

\noindent The University provides some basic guidance about academic integrity: \href{http://sa.berkeley.edu/conduct/integrity}{http://sa.berkeley.\\edu/conduct/integrity}. Lack of knowledge of the academic honesty policy is not a reasonable explanation for a violation. Questions related to course assignments and the academic honesty policy should be directed to me.

\vskip.25in
\noindent My policy is that you may work together on homework, \textit{but you must specifically cite with whom you worked and what you did together}.

\vskip.25in
\noindent\textbf{Extra Help}:  Do not hesitate to come to my office during office hours or by appointment to discuss a homework problem or any aspect of the course. 

\vskip.25in
\noindent\textbf{Attendance}: Students are expected to attend classes regularly. A student who incurs an excessive number of absences may be withdrawn from this class at my discretion.

\vskip.25in
\noindent\textbf{Other Policies}: This course abides by the university policies for
\begin{itemize}
  \item Accommodation of religious creed: \href{http://registrar.berkeley.edu/DisplayMedia.aspx?ID=Religious\%20Creed\%20Policy.pdf}{http://registrar.berkeley.edu/DisplayMedia.aspx?ID\\=Religious\%20Creed\%20Policy.pdf}
  \item Conflicts between extracurricular activities and academic requirements: \href{http://academic-senate.berkeley.edu/sites/default/files/committees/cep/guidelines\_acadschedconflicts\_final\_2014.pdf}{http://academic-senate.berkeley.edu/sites/default/files/committees/cep/guidelines\_acadschedconflicts\\\_final\_2014.pdf}
  \item In case of illness please do not come to class if you have a fever or something highly contagious. Please attend if there is any chance you will pay attention and not get others sick: \href{http://academic-senate.berkeley.edu/committees/coci/toolbox\#16}{http://academic-senate.berkeley.edu/committees/coci/toolbox\#16}
\end{itemize}

\noindent\textbf{Schedule}: \textit{Note that all dates are subject to change}\\

%-----------------------------------------------------------------------------
\clearpage
\begin{tabu}{| c | l | X | c | c |}
\hline
    Lecture & Date & Topic & Assigned & Due \\
    \hline
% Table generated by Excel2LaTeX from sheet '2016'
    1     & 20-Jan & introduction & hw 1  &  \\
    2     & 22-Jan & computing and parallelization &       & hw 1 \\
    3     & 25-Jan & types of equations &       &  \\
    4     & 27-Jan & transport equation &       &  \\
    5     & 29-Jan & transport equation & hw 2  &  \\
    6     & 1-Feb & diffusion equation &       &  \\
    7     & 5-Feb & diffusion equation &       &  \\
    8     & 5-Feb & diffusion equation & hw 3  & hw 2 \\
    9     & 8-Feb & interpolation &       &  \\
    10    & 10-Feb & interpolation cont'd; approximation &       &  \\
    11    & 12-Feb & numerical differentiation & hw 4  & hw 3 \\
    -     & 15-Feb & \textit{President's Day} &       &  \\
    12    & 17-Feb & numerical integration &       &  \\
    13    & 19-Feb & numerical integration &       &  \\
    14    & 22-Feb & vectors and matrices reviews &       &  \\
    15    & 24-Feb & 1-D finite diff and vol intro & hw 5  & hw 4 \\
    16    & 26-Feb & 1-D finite vol for DE &       &  \\
    17    & 29-Feb & norms and convergence &       &  \\
    18    & 2-Mar & direct solvers &       &  \\
    19    & 4-Mar & iterative solvers &       & hw 5 \\
    20    & 7-Mar & catch up + exam review &       &  \\
    21    & 9-Mar & \textbf{Midterm 1 (through TE/DE)} & \textbf{} &  \\
\hline
\end{tabu}%
\begin{tabu}{| c | l | X | c | c |}
\hline
    Lecture & Date & Topic & Assigned & Due \\
    \hline
    22    & 11-Mar & 1-D finite vol soln methods & hw 6  &  \\
    23    & 14-Mar & eigenvalues review &       &  \\
    24    & 16-Mar & eigenvalue solvers &       &  \\
    25    & 18-Mar & FVM for 1-D eigenvalue &       &  \\
    -     & 21-25 Mar & \textit{spring break} & \textit{} &  \\
    26    & 28-Mar & project planning, return midterm, FVM for 1-D eig & hw 7  & hw 6 \\
    27    & 30-Mar & 2-D finite vol for DE &       &  \\
    28    & 1-Apr & 2-D finite vol for DE &       &  \\
    29    & 4-Apr & point kinetics &       &  \\
    30    & 6-Apr & Taylor and Runge Kutta &       & abstract \\
    31    & 8-Apr & predictor-corrector methods & hw 8  & hw 7 \\
    32    & 11-Apr & Monte Carlo intro &       &  \\
    33    & 13-Apr & MC probability and statistics &       &  \\
    34    & 15-Apr & MC random sampling &       &  \\
    35    & 18-Apr & MC tracking and collisions &       & interim report \\
    36    & 20-Apr & MC tallies &       & hw  8 \\
    37    & 22-Apr & exam review, MC wrap up &       &  \\
    38    & 25-Apr & \textbf{Midterm 2 (through MC)} & \textbf{} &  \\
    39    & 27-Apr & variance reduction &       &  \\
    40    & 29-Apr & review midterm 2, go over presentation/report expectations &       &  \\
    -     & 2-6 May & \textit{RRR week} & \textit{} &  \\
    final & 10-May & Final presentations, 7 to 10 pm &       & final reports \\
\hline
\end{tabu}%


%1. ACCOMMODATION OF RELIGIOUS CREED
%In compliance with Education code, Section 92640(a), it is the official policy of the University of California at Berkeley to permit any student to undergo a test or examination, without penalty, at a time when that activity would not violate the student's religious creed, unless administering the examination at an alternative time would impose an undue hardship that could not reasonably have been avoided. Requests to accommodate a student's religious creed by scheduling tests or examinations at alternative times should be submitted directly to the faculty member responsible for administering the examination by the second week of the semester.
%
%Reasonable common sense, judgment and the pursuit of mutual goodwill should result in the positive resolution of scheduling conflicts. The regular campus appeals process applies if a mutually satisfactory arrangement cannot be achieved.
%
%The link to this policy is available in the Religious Creed section of the Academic Calendar webpage.
%
%2. CONFLICTS BETWEEN EXTRACURRICULAR ACTIVITIES AND ACADEMIC REQUIREMENTS
%The Academic Senate has established Guidelines Concerning Scheduling Conflicts with Academic Requirements to address the issue of conflicts that arise between extracurricular activities and academic requirements. These policies specifically concern the schedules of student athletes, student musicians, those with out-of-town interviews, and other students with activities (e.g., classes missed as the result of religious holy days) that compete with academic obligations.
%
%These policies were updated in Spring 2014 to include the following statement:
%
%-The pedagogical needs of the class are the key criteria when deciding whether a proposed accommodation is appropriate. Faculty must clearly articulate the specific pedagogical reasons that prevent accepting a proposed accommodation. Absent such a reason, the presumption should be that accommodations are to be made.
%
%The guidelines assign responsibilities as follows:
%
%-It is the instructor?s responsibility to give students a schedule, available on the syllabus in the first week of instruction, of all class sessions, exams, tests, project deadlines, field trips, and any other required class activities.
%
%-It is the student?s responsibility to notify the instructor(s) in writing by the second week of the semester of any potential conflict(s) and to recommend a solution, with the understanding that an earlier deadline or date of examination may be the most practicable solution.
%
%-It is the student?s responsibility to inform him/herself about material missed because of an absence, whether or not he/she has been formally excused.
%
%The complete guidelines are available on the Academic Senate website. Additionally, a checklist to help instructors and students comply with the guidelines is available on the Center for Teaching and Learning website.
%
%3. ABSENCES DUE TO ILLNESS
%Instructors are asked to refrain from general requirements for written excuses from medical personnel for absence due to illness. Many healthy people experience a mild-to-moderate illness and recover without the need to seek medical attention. University Health Services does not have the capacity to evaluate such illnesses and provide documentation excusing student absences. However, UHS will continue to provide documentation when a student is being treated by Tang for an illness that necessitates a change in course load or an incomplete.
%
%From time-to-time the Academic Senate has issued guidance concerning missed classes and exams due to illnesses such as influenza advising that students not attend class if they have a fever. Should a student experience repeated absences due to illness, it may be appropriate for the faculty member to ask the student to seek medical advice. The Senate guidelines advise faculty to use flexibility and good judgment in determining whether to excuse missed work, extend deadlines, or substitute an alternative assignment. Only the Committee on Courses of Instruction (COCI) can waive the final exam. However, a department chair can authorize an instructor to offer an alternative format for a final exam (e.g., paper, take-home exam) on a one-time basis (http://academic-senate.berkeley.edu/committees/coci/toolbox#16).
%
% 
%
%4. READING, REVIEW, RECITATION (RRR) WEEK
%The Reading, Review, Recitation (RRR) period before final exams provides students time to prepare for exams, to work on papers and projects, and to participate in optional review sessions and meetings with instructors. For the coming semesters, please keep these dates in mind:
%
%In Spring 2015, classes end on Friday, May 1, 2015. RRR week will take place between the last day of classes (May 1) and the first day of the final exam period (Monday, May 11, 2015).
%
%In Fall 2015, classes end on Friday, December 4, 2015. RRR Week will take place between the last day of classes (December 4) and the first day of the final exam period (Monday, December 14, 2015).
%
%Please note that the regular semester classroom will NOT be available during the RRR week unless the instructor requests it through the departmental scheduler.
%
%Presentations of capstone projects, oral presentations, and performances are permitted, although flexibility in scheduling may be required to accommodate students' individual schedules. The introduction of new material is not permitted. Mandatory exams or quizzes and other mandatory activities are also not permitted, with some very limited exceptions (capstone presentations, for example).
%
%Please keep in mind that final exams and papers or projects substituting for final exams may not be due before the final exam week.
%
%Detailed, updated guidelines on RRR week activities are available on the Academic Senate web site. The Office of the Registrar has posted answers to frequently-asked questions about the academic calendar.
%
%In addition, the Center for Teaching and Learning has prepared some suggestions on making RRR week productive for instructors and students. If you have tips or ideas you would like to have added to this page, please email teaching@berkeley.edu.
%
% 
%
%5. COMMENCEMENT CEREMONIES AND FINAL EXAMS
%Campus policy stipulates that graduation ceremonies must take place after the conclusion of final examinations, with the exception of professional school ceremonies with graduate students only.
%
%For Spring 2015, final exams end at 10 pm on Friday, May 15, 2015.
%
%For Fall 2015, final exams end at 10pm on Friday, December 18, 2015
%
%The Spring 2015 Graduates Convocation will be held Saturday May 16, 2015. For more information, please see the Commencement Convocation Events Office website.
%
% 
%
%6. ACCOMMODATION FOR DISABILITY
%Instructors are reminded of their responsibilities for accommodating disabilities in the classroom in the following areas:
%
%Confidentiality:  Information about a student?s disability is confidential, and may not be shared with other students.
%Role of Instructor: Course instructors play a critical role in enabling the University to meet its obligation to appropriately accommodate students with disabilities who are registered with the Disabled Students Program (DSP) and who have been issues a Letter of Accommodation.
%Reading Assignments:
%In advance: Because students with print disabilities usually need assistance from the DSP Alternative Media Center, reading materials should be provided well in advance (two or more weeks) before the reading assignment due date.
%Required or Recommended: Always indicate which course readings (including bCourse postings) are either ?required? readings, or ?recommended.?
%Accessible Format: Reading materials (especially bCourse postings) should be provided in an ?accessible format,? e.g., clearly legible, ?clean? (without stray marks, highlighting, or mark-ups), and whenever possible, in a Word Document or word-searchable PDF.
%For more information about accommodations for students with disabilities, please contact the Disabled Students Program at 510-642-0518 or email DSP Director Paul Hippolitus hippolitus@berkeley.edu. For more information about providing reading assignments in an accessible format, please contact Martha Velasquez directly at dspamc@berkeley.edu.



\end{document}
