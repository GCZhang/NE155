\NeedsTeXFormat{LaTeX2e}
\documentclass[a4paper, 12 pt]{curve}
%\documentclass[12pt]{article}
%\textwidth=7in
%\textheight=9.5in
%\topmargin=-1in
%\headheight=0in
%\headsep=.5in
%\hoffset=-.85in

%\usepackage[cm]{fullpage}
\usepackage[top=0.75in, bottom=0.75in, left=1in, right=1in]{geometry}
\pagestyle{empty}
\usepackage{tabu}
\parindent 0ex
%\usepackage{curve}

\renewcommand{\thefootnote}{\fnsymbol{footnote}}
\begin{document}

\begin{center}
{\bf NE 155\\ Code Final Project Rubric
}
\end{center}

\setlength{\unitlength}{1in}
\begin{picture}(6,.1) 
\put(0,0) {\line(1,0){6.25}}         
\end{picture}

\renewcommand{\arraystretch}{2}

The final paper should be $\sim$6-7 pages per team member. Please include these items in the \underline{final report}:
%
\begin{enumerate}
\item \textbf{Introduction:} What does the code you wrote do? Also preview what you are going to talk about.

\item \textbf{Mathematics:} Write the continuous and discretized equations that you are solving, defining all terms. Include any derivations needed to reach discretized equations as applicable. 

\item \textbf{Algorithms:} Include the algorithms that you implemented in your code.

\item \textbf{Code Use:} Describe how to use your code, including inputs needed and output expected. 

\item \textbf{Test Problems and Results:} Describe any testing you did to demonstrate your code is correct and present any results from test problems.

\item \textbf{References}
\end{enumerate}

\vspace*{1em}
You must submit your code and at least one example input and corresponding output (I strongly encourage you to version control your code and submit access to the repository). Part of the project grade will be based on whether the code executes properly.

\vspace*{2em}
Your \underline{mid-project report} should replace Test Problems and Results section with \textbf{Plans for Completion}, keeping in mind that these plans should include plans for what tests/inputs you will give to your code. The first three sections don't have to be completely polished, but they should at least be very solid drafts. I will provide feedback, so the better they are when I read them the more useful the feedback will be. This should be 4, 6, or 8 pages maximum for 1, 2, or 3 people, respectively (varying depending on compactness of mathematics, algorithms, etc.).

\vspace*{2em}
Please include these same items in your \underline{final presentation}. For projects with one person, aim for 8 minutes and projects with two people aim for 14 minutes. Part of the final presentation should be a code execution demonstration.

\clearpage
I will use the following rubric for evaluating the paper:

\begin{center}
\begin{tabu}{| X | c | c |}\hline
\textbf{Category} & \textbf{Possible Points} & \textbf{Earned Points} \\ \hline \hline
Math, discretizations, and algorithms are correct & 12 & \\ \hline
Code input, output, and tests are well-designed & 12 & \\ \hline
Work implemented correctly & 10 & \\ \hline
Appropriate logical flow of paper & 3 & \\ \hline
Complete sentences; correct grammar and spelling & 8 & \\ \hline
Sources properly documented & 5 & \\ \hline
Total & 50 & \\\hline
\end{tabu} 
\end{center}

\vspace*{1 em}
This rubric is for evaluating the presentation:


\begin{center}
\begin{tabu}{| X | c | c |}\hline
\textbf{Category} & \textbf{Possible Points} & \textbf{Earned Points} \\ \hline \hline
Explanation of implemented code (math, discretizations, algorithms) is clear & 8 & \\ \hline
Code demonstration works and is understandable & 8 & \\ \hline
Good presentation skills: eye contact, volume, clarity of slides, etc. & 6 & \\ \hline
Appropriate presentation length & 3 & \\ \hline
Total & 25 & \\\hline
\end{tabu} 
\end{center}


\end{document}