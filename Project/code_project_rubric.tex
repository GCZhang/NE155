\NeedsTeXFormat{LaTeX2e}
\documentclass[a4paper, 12 pt]{curve}
%\documentclass[12pt]{article}
%\textwidth=7in
%\textheight=9.5in
%\topmargin=-1in
%\headheight=0in
%\headsep=.5in
%\hoffset=-.85in

%\usepackage[cm]{fullpage}
\usepackage[top=0.75in, bottom=0.75in, left=1in, right=1in]{geometry}
\pagestyle{empty}
\usepackage{tabu}
\parindent 0ex
\usepackage{hyperref}

\renewcommand{\thefootnote}{\fnsymbol{footnote}}
\begin{document}

\begin{center}
{\bf NE 155\\ Code Final Project Rubric
}
\end{center}

\setlength{\unitlength}{1in}
\begin{picture}(6,.1) 
\put(0,0) {\line(1,0){6.25}}         
\end{picture}

\renewcommand{\arraystretch}{2}

The final paper should be $\sim$6-7 pages with 1.5 spacing per team member. This may vary based on the specific project; please use your best judgment. Please include these items as clearly labeled sections in the \underline{final report}:
%
\begin{enumerate}
\item \textbf{Introduction:} What does the code you wrote do? Also preview what you are going to talk about.

\item \textbf{Mathematics:} Write the continuous and discretized equations that you are solving, defining all terms. Include any derivations needed to reach discretized equations as applicable. 

\item \textbf{Algorithms:} Include the algorithms that you implemented in your code.

\item \textbf{Code Use:} Describe how to use your code, including inputs needed and output expected. 

\item \textbf{Test Problems and Results:} Describe any testing you did to demonstrate your code is correct and present any results from test problems.

\item \textbf{References:} You must have references that you cite in your paper (including the interim report).
\end{enumerate}

\vspace*{1em}
You must submit your code and at least one example input and corresponding output (I strongly encourage you to version control your code and submit access to the repository). Part of the project grade will be based on whether the code executes properly.

\vspace*{2em}
In your \underline{mid-project report}, please replace the ``Test Problems and Results" section (and possibly also ``Code Use" depending on how far you've gotten) with \textbf{Plans for Completion} and keep in mind that these plans should include plans for what tests/inputs you will give to your code. The first four sections don't have to be completely polished, but they should at least be very solid drafts. I will provide feedback, so the better they are when I read them the more useful the feedback will be. This should be a maximum of 4, 6, or 8 pages with 1.5 spacing for 1, 2, or 3 people, respectively (varying depending on compactness of mathematics, algorithms, etc.).

\vspace*{2em}
Please include these same items in your \underline{final presentation}. The presentations should be approximately 6, 9, or 12 minutes for 1, 2, or 3 people, respectively.  This may vary based on the specific project; please use your best judgment. Part of the final presentation should be a code execution demonstration.

\vspace*{2em}
\textbf{Notes for writing papers properly:}
\begin{itemize}
\item If you include figures, use a Figure number and caption; refer to the figure from within the text as Fig.\ \# or Figure \#.
\item You may need to number equations and refer to them in the text.
\item \textit{Use section headings for the requested sections.}
\item In the introduction, discuss what is coming up in the paper. 
\item In the conclusions, discuss what you told us in the paper.
\item If you talk about a code (that you didn't write yourself), you need to include a reference for that code. 
\item For the final report, it's a good idea to include enough information for the work to be reproducible. To avoid making the report filled with mundane details you can put some items in an appendix.
\item Common grammar errors: \href{http://www.quickanddirtytips.com/education/grammar/which-versus-that-0}{that vs.\ which}, \href{http://grammarpartyblog.com/2012/01/17/use-versus-utilize/}{use vs.\ utilize}, \href{https://e-gmat.com/blog/gmat-verbal/sentence-correction/idioms/due-to-vs-because-of}{due to vs.\ because of}.
\end{itemize}

\vspace*{2em}
I will use the following rubric for evaluating the paper:

\begin{center}
\begin{tabu}{| X | c | c |}\hline
\textbf{Category} & \textbf{Possible Points} & \textbf{Earned Points} \\ \hline \hline
Math, discretizations, and algorithms are correct & 12 & \\ \hline
Code input, output, and tests are well-designed & 12 & \\ \hline
Work implemented correctly & 10 & \\ \hline
Appropriate logical flow of paper & 3 & \\ \hline
Complete sentences; correct grammar and spelling & 8 & \\ \hline
Sources properly documented & 5 & \\ \hline
Total & 50 & \\\hline
\end{tabu} 
\end{center}

\vspace*{1 em}
This rubric is for evaluating the presentation:


\begin{center}
\begin{tabu}{| X | c | c |}\hline
\textbf{Category} & \textbf{Possible Points} & \textbf{Earned Points} \\ \hline \hline
Explanation of implemented code (math, discretizations, algorithms) is clear & 8 & \\ \hline
Code demonstration works and is understandable & 8 & \\ \hline
Good presentation skills: eye contact, volume, clarity of slides, etc. & 6 & \\ \hline
Appropriate presentation length & 3 & \\ \hline
Total & 25 & \\\hline
\end{tabu} 
\end{center}


\end{document}