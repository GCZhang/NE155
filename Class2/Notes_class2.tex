\documentclass[12pt]{article}

\title{NE 155, Class 2, S14}
\author{Rachel Slaybaugh}
\date{January 24, 2014}

\begin{document}
\maketitle

\noident \textbf{How do we measure utility?}

IPS is often specified in the millions, giving MIPS. MIPS is used to measure the integer performance of a computer. Examples of integer operation include data movement (A to B) or value testing (If A = B, then C). MIPS as a performance benchmark is adequate for the computer when it is used in database query, word processing, spreadsheets, or to run multiple virtual operating systems (http://en.wikipedia.org/wiki/FLOPS).

The clock rate of a CPU is normally determined by the frequency of an oscillator crystal. Typically a crystal oscillator produces a fixed sine wave—the frequency reference signal. Electronic circuitry translates that into a square wave at the same frequency for digital electronics applications (or, in using a CPU multiplier, some fixed multiple of the crystal reference frequency). The clock distribution network inside the CPU carries that clock signal to all the parts that need it. (http://en.wikipedia.org/wiki/Clock_rate)

FLOPS measures the computing ability of a computer and includes the concept of clock rate. An example of a floating-point operation is the calculation of mathematical equations; as such, FLOPS is a useful measure of supercomputer performance. This is the term we will use most frequently in this class. 

\end{document}
