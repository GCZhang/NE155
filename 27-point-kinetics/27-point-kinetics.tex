\documentclass[12pt]{article}
\usepackage[top=1in, bottom=1in, left=1in, right=1in]{geometry}

\usepackage{setspace}
\onehalfspacing

\usepackage{amssymb}
%% The amsthm package provides extended theorem environments
\usepackage{amsthm}
\usepackage{epsfig}
\usepackage{times}
\renewcommand{\ttdefault}{cmtt}
\usepackage{amsmath}
\usepackage{graphicx} % for graphics files

% Draw figures yourself
\usepackage{tikz} 

% The float package HAS to load before hyperref
\usepackage{float} % for psuedocode formatting
\usepackage{xspace}

% from Denovo Methods Manual
%\usepackage{mathrsfs}
%\usepackage[mathcal]{euscript}
%\usepackage{color}
%\usepackage{array}

\usepackage[pdftex]{hyperref}
\usepackage[parfill]{parskip}

% math syntax
\newcommand{\nth}{n\ensuremath{^{\text{th}}} }
\newcommand{\ve}[1]{\ensuremath{\mathbf{#1}}}
\newcommand{\Macro}{\ensuremath{\Sigma}}

%---------------------------------------------------------------------------
\title{NE 155, Class 27, S15 \\
Reactor Kinetics in Zero Dimensions}
\date{April 1, 2015}
\begin{document}
\author{guest lecturer: Kathryn Huff}
\maketitle

\hrulefill

\section{Introduction}

In reactors and other fission systems, neutron populations vary over time. This 
lesson will introduce a method for analyzing this time evolution analytically 
by neglecting variation of the flux shape. In particular, this lesson will 
cover:

\begin{itemize}
\item delayed neutrons,
\item the importance of delayed neutrons for reactor control,
\item the derivation of the point reactor kinetics equations (PRKE),
\item and an approach to their solution.
\end{itemize}

Additionally, if we have time, this lesson will also cover feedback effects, 
incuding: 
\begin{itemize}
\item the importance of feedbacks
\item the form of the PRKEs with simple temperature feedback.
\end{itemize}


%------------------------------------------------------------------------------
\section{Delayed Neutrons}
Reactor control relies on a balance of neutrons. When an isotope fissions, it 
produces neutrons, energy, and fission products.  Most of the neutrons emitted 
due to fission are \emph{prompt}, nearly all within $10^{-10}s$ of the fission. 

\subsection{Delayed Neutron Emission}
However, a fraction of the neutros appear later when certain fission products 
emit a neutron upon decay. These certain fission products are called ``delayed 
neutron precursors''.  $^{87}$Br, for example, has a half-life of 55.9 seconds.

\subsection{Delayed Neutron Precursor Data}
Typically, we group delayed neutron precursors into 6 or 8 groups. Standardized 
data exist for these calculations.

    \begin{table}[h!]
    \centering
      \begin{tabular}{|l|c|c|c|c|}
        \hline
        j & $t_{1/2}$ & $\lambda^d_j$  & $\eta_j$ & $\beta_j$\\
          &   $[s]$   &    $[1/s]$     & $[n/f]$  & \\
        \hline
        1   &  $ 55.72 $  &  $ 0.0124 $  &  $ 0.00052 $  &  $ 0.000215$  \\
        2   &  $ 22.72 $  &  $ 0.0305 $  &  $ 0.00546 $  &  $ 0.001424$  \\
        3   &  $ 6.22  $  &  $ 0.111  $  &  $ 0.00310 $  &  $ 0.001274$  \\
        4   &  $ 2.30  $  &  $ 0.301  $  &  $ 0.00624 $  &  $ 0.002568$  \\
        5   &  $ 0.614 $  &  $ 1.14   $  &  $ 0.00182 $  &  $ 0.000748$  \\
        6   &  $ 0.230 $  &  $ 3.01   $  &  $ 0.00066 $  &  $ 0.000273$  \\
        \hline
      \end{tabular}
      \caption{Delayed neutron data, $^{235}$U thermal fission
      \cite{lamarsh_introduction_1975}.}
      \label{tab:delayedneutrons}
    \end{table}


\section{Delayed Neutrons and Reactor Control}
These delayed neutrons are critical to controlling the reactor. If there were 
no delayed neutrons, then the time constant for power increase would simply be 
$l_p$, the prompt neutron lifetime. That isn't the case, but if it were, the 
reactor power would proceed thus:

\begin{align}
P(t) &= P_0k^{\frac{t}{l_p}
\end{align}

In a universe without delayed neutrons, the equation above would be true. 
Noting that the prompt neutron lifetime is about $2\times10^{-5}s$, take a 
moment to think about the implications of this.

\paragraph{Exercise}
If a control rod were moved to introduce an excess reactivity of $0.0005\Delta 
k$, what would the power be one second later?


\section{The Diffusion Equation}
In the steady-state diffusion equation, all neutrons are approxmiated to be ``prompt'' neutrons.

<diffusion eqn>

To incorporate delayed neutrons, the $\chi (E)$ fission spectrum must be properly weighted with prompt and delayed contributions.

We need a time-dependent diffusion equation.

\section{Time Dependence}

<time dep, one speed diffusion>

\section{The Point Reactor Kinetics Equations}

One common method to evaluate transient scenarios is through reduction of
dimensions by the use of of the Point Reactor Kinetics Equations (PRKE). If we assume a separation of variables solution  to \eqref{boltz_one_speed}, we
arrive at:

\begin{align}
  \phi(r,t) &= vn(t)\psi_1(r) \label{sep_var_phi}\\
  \hat{C}_i(r,t) &= C_i(t)\psi_1(r) \label{sep_var_c}\\
  \intertext{where $\psi_1$ is the fundamental mode solution of}
  \nabla^2\psi_n &+ B_g^2\psi_n = 0.
\end{align}

Using this separation of variables solution reduces the spatial complexity of
the reactor to a single point. Inserting \eqref{sep_var_phi} and
\eqref{sep_var_c} into \eqref{boltz_one_speed} gives the Point Reactor Kinetics Equations (PRKE).

\begin{align}
  \frac{dn(t)}{dt} &= \frac{\rho(t)-\beta}{\Lambda}n(t) + \sum_{i=1}^{6}\lambda_iC_i(t)\\
  \frac{dC_i(t)}{dt} &= \frac{\beta_i}{\Lambda}n(t) - \lambda_i C_i(t)
  \label{prke}
\end{align}

\begin{align}
  \intertext{where}
  i&\in[1,6]\nonumber\\
  \Lambda&\equiv (v\nu\Sigma_F)^{-1}\nonumber\\
  \rho(t) &\equiv\frac{k(t)-1}{k(t)}\nonumber\\
         &\equiv\frac{\nu\Sigma_F-\Sigma_a(1+L^2B_g^2)}{\nu\Sigma_F}\nonumber\\
  \intertext{and}
  k&\equiv \frac{\nu\Sigma_F/\Sigma_a}{1+L^2B_g^2}.\nonumber
\end{align}

The PRKEs allow a nuclear engineer to remove the spatial aspects of the
reactor from consideration, thereby only concerning themselves with the integral
flux transients, which manifest as power transients.  In addition to modeling
the neutronic properties of a nuclear reactor, the PRKE can be modified to
include the thermal-hydraulic feedback effects that the power transient will
induce.

The PRKE are a set of stiff, nonlinear ordinary differential equations.
For a reactor in which the only reactivity feedback comes from the fuel and the
coolant:



\begin{equation}
  \frac{d}{dt}\left[
    \begin{array}{c}
      p\\
      \zeta_1\\
      .\\
      .\\
      .\\
      \zeta_j\\
      .\\
      .\\
      .\\
      \zeta_J\\
      \omega_1\\
      .\\
      .\\
      .\\
      \omega_k\\
      .\\
      .\\
      .\\
      \omega_K\\
      T_{fuel}\\
      T_{cool}\\
      T_{refl}\\
      T_{matr}\\
      T_{grph}\\
      .\\
      .\\
      .\\
    \end{array}
    \right]
    =
    \left[
      \begin{array}{ c }
        \frac{\rho(t,T^{fuel},T_{cool},\cdots)-\beta}{\Lambda}p +
        \displaystyle\sum^{j=J}_{j=1}\lambda_j\zeta_j\\
        \frac{\beta_1}{\Lambda} p - \lambda_1\zeta_1\\
        .\\
        .\\
        .\\
        \frac{\beta_j}{\Lambda}p-\lambda_j\zeta_j\\
        .\\
        .\\
        .\\
        \frac{\beta_J}{\Lambda}p-\lambda_J\zeta_J\\
        \kappa_1p - \lambda_1\omega_1\\
        .\\
        .\\
        .\\
        \kappa_kp - \lambda_k\omega_k\\
        .\\
        .\\
        .\\
        \kappa_{k p} - \lambda_k\omega_{k}\\
        f_{fuel}(p, C_p^{fuel}, T_{fuel}, T_{cool},\cdots)\\
        f_{cool}(C_p^{cool}, T_{fuel}, T_{cool},\cdots)\\
        f_{refl}(C_p^{refl}, T_{fuel}, T_{refl},\cdots)\\
        f_{matr}(C_p^{matr}, T_{fuel}, T_{matr},\cdots)\\
        f_{grph}(C_p^{grph}, T_{fuel}, T_{grph},\cdots)\\
        .\\
        .\\
        .\\
      \end{array}
      \right]
      \label{eqn:full_prke}
    \end{equation}



Equation \ref{eqn:full_prke}, shows a generalized set of PRKE where
variables include the normalized power, $p$, the delayed neutron precursor
concentrations $\zeta_j$, decay heats, $\omega_k$, and the core average fuel and
coolant temperatures $T_{fuel}$ and $T_{cool}$.  Additional equations
quantifying other phenomena can add complexity to this suite of PDEs.

\begin{align}
\frac{dn(t)}{dt} &= \frac{\rho(t) - \beta}{l^*}n(t) + \sum_{i=1}^{N} \lambda_i C_i(t) \nonumber \\
\frac{dC_i(t)}{dt} &= \frac{\beta_i}{l^*}n(t) - \lambda_i C_i(t) \qquad i=1,\dots,N \nonumber
\intertext{where}
 n &= \mbox{neutron population}\\
 \beta &= \mbox{fraction of neutrons that are delayed}\\
 \lambda_i &= \mbox{effective decay constant of the ith precursor}[\frac{1}{s}]\\
 C_i(t) &= \mbox{delayed neutron concentration due to the ith precursor}\\
 l &= \mbox{mean neutron lifetime}\\
 \rho &= \mbox{reactivity}\\
 &= \frac{k-1}{k} 
\end{align}

BCs: $n(0) = n_0$ and $C_i(0) = C_{i,0}$ for $i=1,\dots,N$.

\end{document}
