\documentclass[12pt]{article}
\usepackage[top=1in, bottom=1in, left=1in, right=1in]{geometry}

%\usepackage{setspace}
%\onehalfspacing

\usepackage{amssymb}
%% The amsthm package provides extended theorem environments
\usepackage{amsthm}
\usepackage{epsfig}
\usepackage{times}
\renewcommand{\ttdefault}{cmtt}
\usepackage{amsmath}
\usepackage{graphicx} % for graphics files

% Draw figures yourself
\usepackage{tikz} 

% The float package HAS to load before hyperref
\usepackage{float} % for psuedocode formatting
\usepackage{xspace}

% from Denovo Methods Manual
%\usepackage{mathrsfs}
%\usepackage[mathcal]{euscript}
%\usepackage{color}
%\usepackage{array}

\usepackage[pdftex]{hyperref}
\usepackage[parfill]{parskip}

% math syntax
\newcommand{\nth}{n\ensuremath{^{\text{th}}} }
\newcommand{\ve}[1]{\ensuremath{\mathbf{#1}}}
\newcommand{\Macro}{\ensuremath{\Sigma}}

%---------------------------------------------------------------------------
\title{NE 155, Class 27, S15 \\
Reactor Kinetics in Zero Dimensions}
\date{April 1, 2015}
\begin{document}
\author{guest lecturer: Kathryn Huff}
\maketitle

\hrulefill

\section{Introduction}

In reactors and other fission systems, neutron populations vary over time. This 
lesson will introduce a method for analyzing this time evolution analytically 
by neglecting variation of the flux shape. In particular, this lesson will 
cover:

\begin{itemize}
\item delayed neutrons,
\item the importance of delayed neutrons for reactor control,
\item the derivation of the point reactor kinetics equations (PRKE),
\item and an approach to their solution.
\end{itemize}

Additionally, if we have time, this lesson will also cover feedback effects, 
incuding: 
\begin{itemize}
\item the importance of feedbacks
\item the form of the PRKEs with simple temperature feedback.
\end{itemize}

Note that much of this can be found in Duderstadt and Hamilton.

\subsection{Transient Analysis}

Transient analysis is necessary when the neutron flux varies with time.
Commonly studied transient scenarios include normal startup and shutdown of a
reactor as well as abnormal scenarios that cause reactivity increases and
decreases during otherwise normal operation.

%------------------------------------------------------------------------------
\section{Delayed Neutrons}
Reactor control relies on a balance of neutrons. When an isotope fissions, it 
produces neutrons, energy, and fission products.  Most of the neutrons emitted 
due to fission are \emph{prompt}, nearly all within $10^{-10}s$ of the fission. 

\subsection{Delayed Neutron Emission}
However, a fraction of the neutrons appear later. Some fission products are 
unstable and decay within seconds or minutes of the fission. Among those, a few 
decay by neutron emission. These particular fission products are called ``delayed 
neutron precursors''.  $^{87}$Br, for example, has a half-life of 55.9 seconds 
and tends to decay by neutron emsission.

\subsection{Delayed Neutron Precursor Data}
Typically, we group delayed neutron precursors into 6 or 8 groups according to 
their half-lives. Standardized data exist for these calculations, as in Table 
\ref{tab:delayedneutrons}.

    \begin{table}[h!]
    \centering
      \begin{tabular}{|l|c|c|c|c|}
        \hline
        j & $t_{1/2}$ & $\lambda^d_j$  & $\eta_j$ & $\beta_j$\\
          &   $[s]$   &    $[1/s]$     & $[n/f]$  & \\
        \hline
        1   &  $ 55.72 $  &  $ 0.0124 $  &  $ 0.00052 $  &  $ 0.000215$  \\
        2   &  $ 22.72 $  &  $ 0.0305 $  &  $ 0.00546 $  &  $ 0.001424$  \\
        3   &  $ 6.22  $  &  $ 0.111  $  &  $ 0.00310 $  &  $ 0.001274$  \\
        4   &  $ 2.30  $  &  $ 0.301  $  &  $ 0.00624 $  &  $ 0.002568$  \\
        5   &  $ 0.614 $  &  $ 1.14   $  &  $ 0.00182 $  &  $ 0.000748$  \\
        6   &  $ 0.230 $  &  $ 3.01   $  &  $ 0.00066 $  &  $ 0.000273$  \\
        \hline
      \end{tabular}
      \caption{Delayed neutron data, $^{235}$U thermal fission
      \cite{lamarsh_introduction_1975}.}
      \label{tab:delayedneutrons}
    \end{table}


In the above table:
\begin{align}
j &= \mbox{group index}\\
t_{1/} &= \mbox{half life} [s]\\
\lambda_j^d &= \mbox{decay constant} [1/s]\\
\eta_j &= \mbox{fission factor} [n/f]\\
\beta_j &= \mbox{delayed neutron fraction} [\nu_d/\nu_{tot}]
\end{align}

These parameters are used to incorporate the contributions of delayed neutrons 
into transient calcluations.

\section{Delayed Neutrons and Reactor Control}
These delayed neutrons are critical to controlling the reactor. 
To capture the reasons why, we will need the following definitions.

\begin{align}
\rho &= \mbox{reactivity}\\
&= \frac{k-1}{k}\\
k &= \mbox{multiplication factor}\\
&(k < 1) \rightarrow \mbox{large, negative reactivity}\\
&(k > 1) \rightarrow \mbox{large, positive reactivity}\\
&(k < 1) \rightarrow \mbox{critical}\\
\beta &= \mbox{delayed neutron fraction}\\
&(\rho << \beta) \mbox{ controllable transient}\\
&(\rho < \beta) \mbox{ delayed supercriticality}\\
&(\rho > \beta) \mbox{ prompt supercriticality}\\
l &= \mbox{mean neutron lifetime}
\end{align}

\subsection{Units of Reactivity}
Note that the units of $\rho$ can be confusing. 

    \begin{table}[h!]
    \centering
      \begin{tabular}{|l|c|c||}
        \hline
        Unit & Definition & Example\\
        \hline
        $\Delta k$ & actual PRKE units & 0.0005 \\
        $\%\Delta k$ & percent notation of $\Delta k$ & 0.05\% \\
        pcm & per cent mille & 50pcm \\
        \hline
        Dollars & $\frac{\Delta k}{\beta}$ & \$1\\
        Cents & 100 cents per dollar & 100 cents\\
        Milli-beta & 1000 milli-beta per dollar& 1000 milli-beta\\
        \hline
      \end{tabular}
      \caption{Common units of reactivity.}
      \label{tab:rho_units}
    \end{table}


\subsection{Thought Experiment}
If there were no delayed neutrons, then the time constant for power increase would simply be 
$l_p$, the prompt neutron lifetime. That isn't the case, but if it were, the 
reactor power would proceed thus:

\begin{align}
l &= \mbox{mean generation time}\\
n(t+l) &= n(t) + l\frac{dn}{dt} = k_\infty n(t)
\intertext{such that}
\frac{dn}{dt} &= \left(\frac{k-1}{l}\right)n(t)
\intertext{which gives}
n(t) &= n_0e^{\frac{(k-1)t}{l}}
\intertext{characterized by the time constant}
T &= \mbox{reactor period}\\
  &= \frac{l}{k-1}
\end{align}

In a universe without delayed neutrons, the mean neutron lifetime ($l$) would be 
the prompt neutron lifetime ($l_p$).  Noting that the prompt neutron lifetime is about 
$2\times10^{-5}s$, take a moment to think about the implications of this.

\paragraph{Exercise}
\emph{If a control rod were moved to introduce an excess reactivity of }$0.0005\Delta 
k$\emph{, what would the power be one second later?}

\section{Derivation}

It's clear that the delayed neutrons are important. So, how do we include them 
in our model of neutrons in a reactor? 

\subsection{The Diffusion Equation}
Let's begin by observing the steady-state diffusion equation. 

\begin{align}
-\nabla D(\vec{r},E)\nabla\phi(\vec{r},E) +& \Sigma_t(\vec{r},E)\phi(\vec{r},E) 
= \nonumber\\
&\int_0^\infty \Sigma_s(\vec{r},E'\rightarrow E)\phi(\vec{r},E')dE'\nonumber\\ 
&+ \sum_j \chi_T^j(E)\int_0^\infty v\Sigma_p^j(\vec{r},E')\phi(\vec{r},E')dE' + 
Q(\vec{r},E)
\end{align}

Review this equation. Note how, to incorporate delayed neutrons, the $\chi (E)$ 
fission spectrum must be properly weighted with prompt and delayed 
contributions. To incorporate delayed neutrons explicitly (rather than 
implicitly as above), it is necessary to begin with the time-dependent, 
one-speed diffusion equation, and including a source contribution from delayed 
neutrons. That contribution depends on the concentration of delayed neutron 
precursors.  The delayed neutron precursor concentrations obey the equation,

\begin{align}
  \frac{\partial \hat{C}_i(t,r)}{\partial t} =
  \beta_i\nu\Sigma_f(r,t)\phi(r,t) -
  \lambda_i\hat{C}_i(r,t)
  \intertext{where}
  i\in [1,6].
  \label{delayed_precursors}
\end{align}
$\beta_i$ and $\lambda_i$ depend on the incident neutron energy, fissionable
isotope, and precursor group.  In this way, the  linearized Boltzmann transport
equation has seven dimensions.  Taking delayed neutrons into account, the one
speed neutron diffusion equation can be written,

\begin{align}
  \frac{1}{v}\frac{\partial \phi(r,t)}{\partial t} - D(r,t)\nabla^2\phi(r,t) + \
  \Sigma_a(r,t)\phi(r,t) = (1-\beta)\nu\Sigma_F(r,t)\phi(r,t) + \
  \sum_{i=1}^6\lambda_i\hat{C}_i(r,t).
  \label{boltz_one_speed}
\end{align}

Transient analysis methods seek to solve this equation for changes in the
parameters caused by changing reactor conditions. Additional PDEs are added to
this calculation to capture the dependence of temperature on heat conduction and
the effect of fluid flow.

\section{The Point Reactor Kinetics Equations}

One common method to evaluate transient scenarios is through reduction of 
dimensions. If we assume a separation of variables solution  to 
\eqref{boltz_one_speed}, we arrive at:

\begin{align}
  \phi(r,t) &= vn(t)\psi_1(r) \label{sep_var_phi}\\
  \hat{C}_i(r,t) &= C_i(t)\psi_1(r) \label{sep_var_c}\\
  \intertext{where $\psi_1$ is the fundamental mode solution of}
  \nabla^2\psi_n &+ B_g^2\psi_n = 0.
\end{align}

Using this separation of variables solution reduces the spatial complexity of
the reactor to a single point. Inserting \eqref{sep_var_phi} and
\eqref{sep_var_c} into \eqref{boltz_one_speed} gives the Point Reactor Kinetics Equations (PRKE).

\begin{align}
  \frac{dn(t)}{dt} &= \frac{\rho(t)-\beta}{\Lambda}n(t) + \sum_{i=1}^{6}\lambda_iC_i(t)\\
  \frac{dC_i(t)}{dt} &= \frac{\beta_i}{\Lambda}n(t) - \lambda_i C_i(t)
  \label{prke}
  \intertext{where}
  i&\in[1,6]\nonumber\\
  n &= \mbox{neutron population}\\
  \beta &= \mbox{fraction of neutrons that are delayed}\\
  \lambda_i &= \mbox{effective decay constant of the ith precursor}[\frac{1}{s}]\\
  C_i(t) &= \mbox{delayed neutron concentration due to the ith precursor}\\
  l &= \mbox{mean neutron lifetime}\\
  \Lambda &= \mbox{prompt neutron lifetime}\\
  &\equiv (v\nu\Sigma_F)^{-1}\nonumber\\
  \rho(t) \rho &= \mbox{reactivity}\\
  &\equiv\frac{k(t)-1}{k(t)}\nonumber\\
         &\equiv\frac{\nu\Sigma_F-\Sigma_a(1+L^2B_g^2)}{\nu\Sigma_F}\nonumber\\
  \intertext{and}
  k&\equiv \frac{\nu\Sigma_F/\Sigma_a}{1+L^2B_g^2}.\nonumber
\end{align}

The PRKEs allow a nuclear engineer to remove the spatial aspects of the
reactor from consideration to simplify the analysis of this initial value 
problem. In particular, the assumptions that have gone into this set of 
equations include:

\begin{itemize}
\item there is no external neutron source
\item Separation of variables ($\phi(\vec{r},E,t) = S(\vec{r},E,t)T(\vec{r},E,t)$)
\item $\beta_i$ and $\lambda_i$ are constant over the transient
\item There is no feedback (e.g. flux is low, feedbacks are slow/weak, $\rho(t)$ is known)
\end{itemize}

%BCs: $n(0) = n_0$ and $C_i(0) = C_{i,0}$ for $i=1,\dots,N$.

\paragraph{Exercise:}
\emph{What initial conditions need to be defined in order to solve this initial 
value problem (eqn. \eqref{prke}?}
\section{Coupling to Thermal-Hydraulic Feedbacks}

In addition to modeling
the neutronic properties of a nuclear reactor, the PRKE can be modified to
include the thermal-hydraulic feedback effects that the power transient will
induce.

\paragraph{Exercise:} \emph{Can you think of some sources of feedbak in a 
reacor?}

\subsection{Coupled Multiphysics}

Typcially, transient analyses in reactors seek to characterize the relationship
between neutron population and temperature, which are coupled together by
reactivity.  That is, any change in power of a reactor can be related to a
quantity known as the reactivity, $\rho$, which characterizes the offset of the
nuclear reactor from criticality. In all power reactors, the the scalar flux of
neutrons is what determines the reactor's power. The reactor power, in turn,
affects the temperature. Reactivity feedback then results, due the temperature
dependence of geometry, material densities, the neutron spectrum, and
microscopic cross sections \cite{bell_nuclear_1970}.

Nuclear reactors operate in state of criticality, which implies a steady state
neutron flux (and hence power).  Should the reactor deviate from criticality, a
power transient will result. The magnitude and duration of the power transient
will be dependent upon the length and strength of the deviation.  In an
Accident Transient Without Scram (ATWS), only intrinsic negative feedback 
responses within a reactor design
are relied upon to prevent an uncontrollable power excursion.  Since some
feedback may be positive while other feedback is negative, an analysis of the
balance must be addressed. Additionally, due to the effect of delayed neutrons,
the timescales of fluid flow, heat conduction, thermal transport, or other
phenomena, both positive and negative feedback may be delayed. Thus, a transient
assessment of the time evolution of the neutron population within the core is
essential to capture power-level instabilities resulting when reactivity
feedback that is out-of-phase with the neutron population
\cite{stacey_nuclear_2007}.

\subsection{Coupled PRKE}
The PRKE are a set of stiff, nonlinear ordinary differential equations.
For a reactor in which the only reactivity feedback comes from the fuel and the
coolant:



\begin{equation}
  \frac{d}{dt}\left[
    \begin{array}{c}
      p\\
      \zeta_1\\
      .\\
      .\\
      .\\
      \zeta_j\\
      .\\
      .\\
      .\\
      \zeta_J\\
      \omega_1\\
      .\\
      .\\
      .\\
      \omega_k\\
      .\\
      .\\
      .\\
      \omega_K\\
      T_{fuel}\\
      T_{cool}\\
      T_{refl}\\
      T_{matr}\\
      T_{grph}\\
      .\\
      .\\
      .\\
    \end{array}
    \right]
    =
    \left[
      \begin{array}{ c }
        \frac{\rho(t,T^{fuel},T_{cool},\cdots)-\beta}{\Lambda}p +
        \displaystyle\sum^{j=J}_{j=1}\lambda_j\zeta_j\\
        \frac{\beta_1}{\Lambda} p - \lambda_1\zeta_1\\
        .\\
        .\\
        .\\
        \frac{\beta_j}{\Lambda}p-\lambda_j\zeta_j\\
        .\\
        .\\
        .\\
        \frac{\beta_J}{\Lambda}p-\lambda_J\zeta_J\\
        \kappa_1p - \lambda_1\omega_1\\
        .\\
        .\\
        .\\
        \kappa_kp - \lambda_k\omega_k\\
        .\\
        .\\
        .\\
        \kappa_{k p} - \lambda_k\omega_{k}\\
        f_{fuel}(p, C_p^{fuel}, T_{fuel}, T_{cool},\cdots)\\
        f_{cool}(C_p^{cool}, T_{fuel}, T_{cool},\cdots)\\
        f_{refl}(C_p^{refl}, T_{fuel}, T_{refl},\cdots)\\
        f_{matr}(C_p^{matr}, T_{fuel}, T_{matr},\cdots)\\
        f_{grph}(C_p^{grph}, T_{fuel}, T_{grph},\cdots)\\
        .\\
        .\\
        .\\
      \end{array}
      \right]
      \label{eqn:full_prke}
    \end{equation}



Equation \ref{eqn:full_prke}, shows a generalized set of PRKE where
variables include the normalized power, $p$, the delayed neutron precursor
concentrations $\zeta_j$, and the core average fuel and
coolant temperatures $T_{fuel}$ and $T_{cool}$.  Additional equations
quantifying other phenomena can add complexity to this suite of PDEs.



\end{document}
