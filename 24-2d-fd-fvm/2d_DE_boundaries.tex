%--------------------------------------------------------------------
% NE 155 (intro to numerical simulation of radiation transport)
% Spring 2015

% Exam Template from UMTYMP and Math Department courses
%
% Using Philip Hirschhorn's exam.cls: http://www-math.mit.edu/~psh/#ExamCls
%
% run pdflatex on a finished exam at least three times to do the grading table on front page.
%
%%%%%%%%%%%%%%%%%%%%%%%%%%%%%%%%%%%%%%%%%%%%%%%%%%%%%%%%%%%%%%%%%%%%%%%%%%%%%%%%%%%%%%%%%%%%%%

% These lines can probably stay unchanged, although you can remove the last
% two packages if you're not making pictures with tikz.
\documentclass[12pt]{article}
\RequirePackage{amssymb, amsfonts, amsmath, latexsym, verbatim, xspace, setspace}
\RequirePackage{tikz, pgflibraryplotmarks}
\usetikzlibrary{arrows}

% By default LaTeX uses large margins.  This doesn't work well on exams; problems
% end up in the "middle" of the page, reducing the amount of space for students
% to work on them.
\usepackage[margin=1in]{geometry}
\usepackage{enumerate}
\usepackage{graphicx}

\newcommand{\nth}{n\ensuremath{^{\text{th}}} }
\newcommand{\ve}[1]{\ensuremath{\mathbf{#1}}}
\newcommand{\Macro}{\ensuremath{\Sigma}}
\newcommand{\vOmega}{\ensuremath{\hat{\Omega}}}

% For an exam, single spacing is most appropriate
\singlespacing
% \onehalfspacing
% \doublespacing

% For an exam, we generally want to turn off paragraph indentation
\parindent 0ex

\begin{document} 

\begin{flushright}
\begin{tabular}{p{5in} r l}
NE 155 & Spring 2016 \\
2D DE Reflecting Boundaries
\end{tabular}
\end{flushright}
\rule[1ex]{\textwidth}{.1pt}

%%%%%%%%%%%%%%%%%%%%%%%%%%%%%%%%%%%%%%%%%%%%%%%%%%%%%%%%%%%%%%%%%%%%%%%%%%%%%%%%%%%%%
%
% See http://www-math.mit.edu/~psh/#ExamCls for full documentation, but the questions
% below give an idea of how to write questions [with parts] and have the points
% tracked automatically on the cover page.
%
%
%%%%%%%%%%%%%%%%%%%%%%%%%%%%%%%%%%%%%%%%%%%%%%%%%%%%%%%%%%%%%%%%%%%%%%%%%%%%%%%%%%%%%


Derive the matrix expressions each reflecting boundary in the 2-D diffusion equation formulation derived with the finite volume method and describe the associated equations. 

In 2-D:
\[-\frac{\partial}{\partial x}D(x,y)\frac{\partial}{\partial x} \phi(x,y) -\frac{\partial}{\partial y}D(x,y)\frac{\partial}{\partial y} \phi(x,y) + \Sigma_a(x,y) \phi(x,y) = S(x,y)\]
%
with $x \in [0,a]$ and $y \in [0,b]$.


%-----------------------------------------------------------------------
\subsection*{LEFT}
The left side boundary condition is:
\[\frac{\partial }{\partial x} \phi(x,y)\big|_{x=0} = 0\]

\begin{minipage}{0.45\textwidth}
\begin{tikzpicture}
% grid
\draw[step=2cm] (0,0) grid (2,4);
% volume labels
\node at (1,1) {$V_{1,j}$};
\node at (1,3) {$V_{1,j+1}$};
% surface labels
\node[left] at (-.25,3) {$s_8$};
\node[right] at (2,3) {$s_5$};
\node[left] at (-.25,1) {$s_1$};
\node[right] at (2,1) {$s_4$};
\node[above] at (1,4) {$s_6$};
\node[below] at (1,0) {$s_3$};
% surrounding grid and markers
\draw plot[mark=*, mark options={fill=black}] (0,2) node[left] {$(0,j)$};
\draw (2,2) -- plot[mark=*, mark options={fill=black}] (4,2) node[right] {$(1,j)$};
\draw (0,0)-- plot[mark=*, mark options={fill=black}] (0,-2) node[below] {$(0,j-1)$};
\draw (0,4)-- plot[mark=*, mark options={fill=black}] (0,6) node[above] {$(0,j+1)$};
% arrows
\draw[<->](3.5,2) -- (3.5,6);
\node[right] at (3.5,4) {$\epsilon_{j+1}$};
\draw[<->](3.5,-2) -- (3.5,2);
\node[right] at (3.5,0) {$\epsilon_{j}$};
\draw[<->](0, -1.5) -- (4,-1.5);
\node[below] at (2, -1.5) {$\delta_{1}$};
\end{tikzpicture}
\end{minipage}
%
\begin{minipage}{0.5\textwidth}
To implement the boundary condition, we integrate in the first half-cell, so from $x =0$ to $x=\delta_1 / 2$. 

Recall that when we considered the streaming term we converted the volume integral into a surface integral: 
\begin{align}
&-\int_V d\vec{r}\:\bigl[\nabla \cdot \bigl(D(\vec{r})\nabla \phi(\vec{r})\bigr)\bigr] \nonumber \\
 &= -\int_S d\vec{S} \:D(\vec{r})\frac{\partial}{\partial \hat{n}}\phi(\vec{r}) \nonumber
\end{align}
%
and we defined the partial derivative w.r.t.\ direction on each surface. We now have two fewer surfaces (no $s_2$ or $s_7$), and we apply the zero current boundary condition to $s_1$ and $s_8$.
\end{minipage}

\begin{align}
\frac{\partial}{\partial \hat{n}}\phi(\vec{r}) &= \frac{\phi_{0,j-1} - \phi_{0,j}}{\epsilon_j} \qquad \text{on } S_3 \nonumber \\
%
&= \frac{\phi_{0,j+1} - \phi_{0,j}}{\epsilon_{j+1}} \qquad \text{on } S_6 \nonumber \\
%
&= \frac{\phi_{1,j} - \phi_{0,j}}{\delta_{1}} \qquad \text{on } S_4 \:, S_5 \nonumber \\
%
&= 0 \qquad \qquad \qquad \text{on } S_1 \:, S_8 \nonumber 
\end{align}
%
We then use the midpoint rule for the integration and integrate along each surface. Recall that the physics values are cell-centered while the flux is edge-centered. The two terms that are different are for the top and bottom of the cell:
%
\begin{align*}
- \int_{S_3} d\vec{S} \:D(\vec{r})\frac{\partial}{\partial \hat{n}}\phi(\vec{r}) &= \frac{\phi_{0,j} - \phi_{0,j-1}}{\epsilon_{j}} \biggl(\frac{D_{1,j} \delta_{1}}{2}\biggr) \: \\ 
%
- \int_{S_6} d\vec{S} \:D(\vec{r})\frac{\partial}{\partial \hat{n}}\phi(\vec{r}) &= \frac{\phi_{0,j+1} - \phi_{0,j}}{\epsilon_{j+1}} \biggl(\frac{D_{1,j+1} \delta_{1}}{2}\biggr) \:\\
%
S_1+S_8 &= 0\:,\\
%
S_4+S_5 &= \frac{\phi_{1,j} - \phi_{0,j}}{\delta_{1}} \biggl(\frac{D_{1,j} \epsilon_{j} + D_{1,j+1} \epsilon_{j+1}}{2}\biggr)\:.
\end{align*}

The absorption and streaming terms now become:
\begin{align}
\int \int dx dy\:\Sigma_a(x,y) \phi(x,y) &= \boxed{\phi_{0,j}\bigl(\Sigma_{a,1,j} V_{1,j} + \Sigma_{a,1,j+1} V_{1,j+1} \bigr) \equiv \Sigma^*_{a,0j}}\:, \nonumber \\
%
\int \int dx dy \: \:S(x,y) &= \boxed{S_{1,j} V_{1,j} + S_{1,j+1} V_{1,j+1} \equiv S^*_{0j}}\:. \nonumber
\end{align}

%-----------------------------------
\vspace*{2em}
Collecting all of the terms and separating them, we get a 4-point difference equation for $i=0$; $j=1,\dots,m-1$:
%
\[\boxed{a_{1,j}^{0j}\phi_{1,j} + a_{0,j-1}^{*,0j}\phi_{0,j-1} + a_{0,j+1}^{*,0j}\phi_{0,j+1} +  a_{0,j}^{*,0j}\phi_{0,j} = S^*_{0j}} \]
%
Recall: the lower index is the cell to which you are coupling, and the upper index is which cell you are in. The coefficients how change to become
\begin{align}
a_{1,j}^{0j} &= -\frac{D_{1,j} \epsilon_{j} + D_{1,j+1} \epsilon_{j+1}}{2 \delta_{1}}  \nonumber \\
%
a_{0,j-1}^{*,0j} &= -\frac{D_{1,j} \delta_{1}}{2 \epsilon_{j}}  \nonumber \\
%
a_{0,j+1}^{*,0j} &= -\frac{D_{1,j+1} \delta_{1}}{2 \epsilon_{j+1}}  \nonumber \\
%
a_{0,j}^{*,0j} &= \Sigma^*_{a,0j} - \bigl(a_{1,j}^{0j} + a_{0,j-1}^{*,0j} + a_{0,j+1}^{*,0j} \bigr)
 \:.\nonumber 
\end{align}



%-----------------------------------------------------------------------
%\subsection*{RIGHT}
%The right side boundary condition is:
%\[\frac{\partial }{\partial x} \phi(x,y)\big|_{x=a} = 0\]
%
%\begin{minipage}{0.45\textwidth}
%\begin{tikzpicture}
%% grid
%\draw[step=2cm] (2,0) grid (4,4);
%% volume labels
%\node at (3,1) {$V_{n,j}$};
%\node at (3,3) {$V_{n,j+1}$};
%% surface labels
%\node[left] at (2,3) {$s_8$};
%\node[right] at (4,3) {$s_5$};
%\node[left] at (2,1) {$s_1$};
%\node[right] at (4,1) {$s_4$};
%\node[above] at (3,4) {$s_7$};
%\node[below] at (3,0) {$s_2$};
%% surrounding grid and markers
%\draw plot[mark=*, mark options={fill=black}] (0,2) node[left] {$(n-1,j)$};
%\draw (0,2) -- plot[mark=*, mark options={fill=black}] (4,2) node[right] {$(n,j)$};
%\draw (4,0)-- plot[mark=*, mark options={fill=black}] (4,-2) node[below] {$(n,j-1)$};
%\draw (4,4)-- plot[mark=*, mark options={fill=black}] (4,6) node[above] {$(n,j+1)$};
%% arrows
%\draw[<->](0.5,2) -- (0.5,6);
%\node[left] at (0.5,4) {$\epsilon_{j+1}$};
%\draw[<->](0.5,-2) -- (0.5,2);
%\node[left] at (0.5,0) {$\epsilon_{j}$};
%\draw[<->](0, -1.5) -- (4,-1.5);
%\node[below] at (2, -1.5) {$\delta_{n}$};
%\end{tikzpicture}
%\end{minipage}
%%
%\begin{minipage}{0.5\textwidth}
%To implement the boundary condition, we integrate in the last half-cell, so from $x=\delta_n / 2$ to $x=a$. 
%
%Recall that when we considered the streaming term we converted the volume integral into a surface integral: 
%\begin{align}
%&-\int_V d\vec{r}\:\bigl[\nabla \cdot \bigl(D(\vec{r})\nabla \phi(\vec{r})\bigr)\bigr] \nonumber \\
% &= -\int_S d\vec{S} \:D(\vec{r})\frac{\partial}{\partial \hat{n}}\phi(\vec{r}) \nonumber
%\end{align}
%%
%and we defined the partial derivative w.r.t.\ direction on each surface. We now have two fewer surfaces (no $s_3$ or $s_6$), and we apply the zero current boundary condition to $s_4$ and $s_5$.
%\end{minipage}
%
%\begin{align}
%\frac{\partial}{\partial \hat{n}}\phi(\vec{r}) &= \frac{\phi_{n,j-1} - \phi_{n,j}}{\epsilon_j} \qquad \text{on } S_2  \nonumber \\
%%
%&= \frac{\phi_{n,j+1} - \phi_{n,j}}{\epsilon_{j+1}} \qquad \text{on } S_7 \nonumber \\
%%
%&= \frac{\phi_{n-1,j} - \phi_{n,j}}{\delta_{n}} \qquad \text{on } S_1 \:, S_8 \nonumber \\
%%
%&= 0 \qquad \qquad \qquad \text{on } S_4 \:, S_5 \nonumber 
%\end{align}
%%
%We then use the midpoint rule for the integration and integrate along each surface. Recall that the physics values are cell-centered while the flux is edge-centered. The two terms that are different are for the top and bottom of the cell:
%%
%\begin{align}
%- \int_{S_2} d\vec{S} \:D(\vec{r})\frac{\partial}{\partial \hat{n}}\phi(\vec{r}) &= \frac{\phi_{n,j} - \phi_{n,j-1}}{\epsilon_{j}} \biggl(\frac{D_{n,j} \delta_{n}}{2}\biggr) \nonumber \\ 
%%
%- \int_{S_7} d\vec{S} \:D(\vec{r})\frac{\partial}{\partial \hat{n}}\phi(\vec{r}) &= \frac{\phi_{n,j+1} - \phi_{n,j}}{\epsilon_{j+1}} \biggl(\frac{D_{n,j+1} \delta_{n}}{2}\biggr) \nonumber
%\end{align}
%
%The absorption and streaming terms now become:
%\begin{align}
%\int \int dx dy\:\Sigma_a(x,y) \phi(x,y) &= \boxed{\phi_{n,j}\bigl(\Sigma_{a,n,j} V_{n,j} + \Sigma_{a,n,j+1} V_{n,j+1} \bigr) \equiv \Sigma^*_{a,nj}}\:, \nonumber \\
%%
%\int \int dx dy \: \:S(x,y) &= \boxed{S_{n,j} V_{n,j} + S_{n,j+1} V_{n,j+1} \equiv S^*_{nj}}\:. \nonumber
%\end{align}
%
%%-----------------------------------
%\vspace*{2em}
%Collecting all of the terms and separating them, we get a 4-point difference equation for $i=n$; $j=1,\dots,m-1$:
%%
%\[\boxed{a_{n-1,j}^{*,nj}\phi_{n-1,j} + a_{n,j-1}^{*,nj}\phi_{n,j-1} + a_{n,j+1}^{*,nj}\phi_{n,j+1} +  a_{n,j}^{*,nj}\phi_{n,j} = S^*_{nj}} \]
%%
%Recall: the lower index is the cell to which you are coupling, and the upper index is which cell you are in. The coefficients how change to become
%%
%\begin{align}
%a_{n-1,j}^{*,nj} &= -\frac{D_{n,j} \epsilon_{j} + D_{n,j+1} \epsilon_{j+1}}{2 \delta_{n}}  \nonumber \\
%%
%a_{n,j-1}^{*,nj} &= -\frac{D_{n,j} \delta_{n}}{2 \epsilon_{j}}  \nonumber \\
%%
%a_{n,j+1}^{*,nj} &= -\frac{D_{n,j+1} \delta_{n}}{2 \epsilon_{j+1}}  \nonumber \\
%%
%a_{n,j}^{*,nj} &= \Sigma_{a,nj} - \bigl(a_{n-1,j}^{*,nj} + a_{n,j-1}^{*,nj} + a_{n,j+1}^{*,nj} \bigr)
% \:.\nonumber
%\end{align}


%-----------------------------------------------------------------------
\newpage
\subsection*{TOP}
The top boundary condition is:
\[\frac{\partial }{\partial y} \phi(x,y)\big|_{y=b} = 0\]

\begin{tikzpicture}
% grid
\draw[step=2cm] (2,2) grid (6,4);
% volume labels
\node at (3,3) {$V_{i,m}$};
\node at (5,3) {$V_{i+1,m}$};
% surface labels
\node[left] at (2,3) {$s_1$};
\node[below] at (3,2) {$s_2$};
\node[below] at (5,2) {$s_3$};
\node[right] at (6,3) {$s_4$};
\node[above] at (5,4) {$s_6$};
\node[above] at (3,4) {$s_7$};
% surrounding grid and markers
\draw plot[mark=*, mark options={fill=black}] (0,4) node[above] {$(i-1,m)$};
\draw (0,4) -- plot[mark=*, mark options={fill=black}] (4,4) node[above] {$(i,m)$};
\draw (4,4)-- plot[mark=*, mark options={fill=black}] (4,0) node[below] {$(i,m-1)$};
\draw (4,4)-- plot[mark=*, mark options={fill=black}] (8,4) node[above] {$(i+1,m)$};
% arrows
\draw[<->](0,0.5) -- (4,0.5);
\node[below] at (2,0.5) {$\delta_{i}$};
\draw[<->](4,0.5) -- (8,0.5);
\node[below] at (6,0.5) {$\delta_{i+1}$};
\draw[<->](0.5, 0) -- (0.5,4);
\node[left] at (0.5,2) {$\epsilon_{m}$};
\end{tikzpicture}

To implement the boundary condition, we integrate in the last half-cell, so from $y=b-\epsilon_m / 2$ to $y=b$. 

Recall that when we considered the streaming term we converted the volume integral into a surface integral: 
\begin{align}
&-\int_V d\vec{r}\:\bigl[\nabla \cdot \bigl(D(\vec{r})\nabla \phi(\vec{r})\bigr)\bigr] \nonumber \\
 &= -\int_S d\vec{S} \:D(\vec{r})\frac{\partial}{\partial \hat{n}}\phi(\vec{r}) \nonumber
\end{align}
%
and we defined the partial derivative w.r.t.\ direction on each surface. We now have two fewer surfaces (no $s_8$ or $s_5$), and we apply the zero current boundary condition to $s_7$ and $s_6$.


\begin{align}
\frac{\partial}{\partial \hat{n}}\phi(\vec{r}) &= \frac{\phi_{i-1,m} - \phi_{i,m}}{\delta_{i}} \qquad \text{on } S_1  \nonumber \\
%
&= \frac{\phi_{i+1,m} - \phi_{i,m}}{\delta_{i+1}} \qquad \text{on } S_4 \nonumber \\
%
&= \frac{\phi_{i,m-1} - \phi_{i,m}}{\epsilon_m} \qquad \text{on } S_2 \:, S_3 \nonumber \\
%
&= 0 \qquad \qquad \qquad \text{on } S_7 \:, S_6 \nonumber 
\end{align}

We then use the midpoint rule for the integration and integrate along each surface. Recall that the physics values are cell-centered while the flux is edge-centered. The two terms that are different are for the left and right of the cell:
%
\begin{align}
- \int_{S_1} d\vec{S} \:D(\vec{r})\frac{\partial}{\partial \hat{n}}\phi(\vec{r}) &= \frac{\phi_{i,m} - \phi_{i-1,m}}{\delta_{i}} \biggl(\frac{D_{i,m} \epsilon_{m}}{2}\biggr) \nonumber \\ 
%
- \int_{S_4} d\vec{S} \:D(\vec{r})\frac{\partial}{\partial \hat{n}}\phi(\vec{r}) &= \frac{\phi_{i+1,m} - \phi_{i,m}}{\delta_{i+1}} \biggl(\frac{D_{i+1,m} \epsilon_{m}}{2}\biggr) \nonumber
\end{align}

The absorption and streaming terms now become:
\begin{align}
\int \int dx dy\:\Sigma_a(x,y) \phi(x,y) &= \boxed{\phi_{n,j}\bigl(\Sigma_{a,i,m} V_{i,m} + \Sigma_{a,i+1,m} V_{i+1,m} \bigr)  \equiv \Sigma^*_{a,im}}\:, \nonumber \\
%
\int \int dx dy \: \:S(x,y) &= \boxed{S_{i,m} V_{i,m} + S_{i+1,m} V_{i+1,m} \equiv S^*_{im}}\:. \nonumber
\end{align}

%-----------------------------------
\vspace*{2em}
Collecting all of the terms and separating them, we get a 4-point difference equation for $i=1,\dots,n-1$, $j=m$:
%
\[\boxed{a_{i-1,m}^{*,im}\phi_{i-1,m} + a_{i,m-1}^{*,im}\phi_{i,m-1} + a_{i+1,m}^{*,im}\phi_{i+1,m} +  a_{i,m}^{*,im}\phi_{i,m} = S^*_{im}} \]
%
Recall: the lower index is the cell to which you are coupling, and the upper index is which cell you are in. The coefficients how change to become
%
\begin{align}
a_{i-1,m}^{*,im} &= -\frac{D_{i,m} \epsilon_{m}}{2\delta_{i}}  \nonumber \\
%
a_{i,m-1}^{*,im} &= -\frac{D_{i,m} \delta_{i} + D_{i+1,m} \delta_{i+1}}{2 \epsilon_{m}}  \nonumber \\
%
a_{i+1,m}^{*,im} &= -\frac{D_{i+1,m} \epsilon_{m}}{2\delta_{i+1}} \nonumber \\
%
a_{n,j}^{*,im} &= \Sigma_{a,im} - \bigl(a_{i-1,m}^{*,im} + a_{i,m-1}^{*,im} + a_{i+1,m}^{*,im} \bigr)
 \:.\nonumber
\end{align}



\end{document}
