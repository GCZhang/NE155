\documentclass[12pt]{article}
\usepackage[top=1in, bottom=1in, left=1in, right=1in]{geometry}

\usepackage{setspace}
\onehalfspacing

\usepackage{amssymb}
%% The amsthm package provides extended theorem environments
\usepackage{amsthm}
\usepackage{epsfig}
\usepackage{times}
\renewcommand{\ttdefault}{cmtt}
\usepackage{amsmath}
\usepackage{graphicx} % for graphics files

% Draw figures yourself
\usepackage{tikz} 

% The float package HAS to load before hyperref
\usepackage{float} % for psuedocode formatting
\usepackage{xspace}

% from Denovo Methods Manual
%\usepackage{mathrsfs}
%\usepackage[mathcal]{euscript}
%\usepackage{color}
%\usepackage{array}

\usepackage[pdftex]{hyperref}
\usepackage[parfill]{parskip}

% math syntax
\newcommand{\nth}{n\ensuremath{^{\text{th}}} }
\newcommand{\ve}[1]{\ensuremath{\mathbf{#1}}}
\newcommand{\Macro}{\ensuremath{\Sigma}}

%---------------------------------------------------------------------------
\title{NE 155, Class 28, S15 \\
Taylor Series Methods and Runge-Kutta}
\date{April 3, 2015}
\begin{document}
\author{guest lecturer: Kathryn Huff}
\maketitle

\hrulefill

\section{Introduction}

LeVeque book section

%------------------------------------------------------------------------------
\section{Taylor Series Derivation of Forward Euler}

Simplest method, Forward euler, can be derived with a taylor series expansion 
around a point in a finite difference system. 

<insert>

\section{Runge-Kutta Methods}

To avoid multistep methods \dots
Consider two stage explicit Runge-Kutta


\section{Application to PRKE}
Each of these can be applied to the PRKE\dots


\section{One-Step vs. MultiStep}
Introduce the notion of multistep methods.
Consider the advantages of one-step over multistep.

\end{document}
