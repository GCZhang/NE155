%--------------------------------------------------------------------
% NE 155 (intro to numerical simulation of radiation transport)
% Linear Algebra Lecture
% Spring 2014

% formatting
\documentclass[12pt]{article}
\usepackage[top=1in, bottom=1in, left=1in, right=1in]{geometry}

\usepackage{setspace}
\onehalfspacing

\setlength{\parindent}{0mm} \setlength{\parskip}{1em}


% packages
\usepackage{amssymb}
%% The amsthm package provides extended theorem environments
\usepackage{amsthm}
\usepackage{epsfig}
\usepackage{times}
\renewcommand{\ttdefault}{cmtt}
\usepackage{amsmath}
\usepackage{graphicx} % for graphics files

% Draw figures yourself
\usepackage{tikz} 

% The float package HAS to load before hyperref
\usepackage{float} % for psuedocode formatting
\usepackage{xspace}

% from Denovo Methods Manual
\usepackage{mathrsfs}
\usepackage[mathcal]{euscript}
\usepackage{color}
\usepackage{array}

\usepackage[pdftex]{hyperref}

\newcommand{\nth}{n\ensuremath{^{\text{th}}} }
\newcommand{\ve}[1]{\ensuremath{\mathbf{#1}}}
\newcommand{\Macro}{\ensuremath{\Sigma}}
\newcommand{\vOmega}{\ensuremath{\hat{\Omega}}}


%--------------------------------------------------------------------
%--------------------------------------------------------------------
\begin{document}
\begin{center}
{\bf NE 155, Class 4 and 5, S14 \\
Types of Equations cont'd + TE + DE \\ January 29 and 31, 2014}
\end{center}

\setlength{\unitlength}{1in}
\begin{picture}(6,.1) 
\put(0,0) {\line(1,0){6.25}}         
\end{picture}

%-------------------------------------------------------------
\noindent \textbf{Clarification on PDE Classification}
%http://en.wikipedia.org/wiki/Partial_differential_equation#Classification

\begin{equation}
A\frac{\partial^2 u}{\partial x^2} + B\frac{\partial^2 u}{\partial x \partial  y} + C\frac{\partial^2 u}{\partial y^2} + \text{ Lower Order Terms } = 0 \nonumber
\end{equation}
%
This equation is a 2nd order PDE in two variables. It is linear if $A$ through $G$ do not depend on $u$ (they may depend on $x$ and $y$. If $A^2 + B^2 + C^2 > 0$ over a region of the $xy$ plane, the PDE is second-order in that region (analogous to the eqn for a conic section: $Ax^2 + Bxy + Cy^2 + \cdots = 0$. 

Replacing $\partial x$ by $x$, and likewise for other variables (formally this is done by a Fourier transform), converts a constant-coefficient PDE into a polynomial of the same degree, with the top degree (a homogeneous polynomial, here a quadratic form) being most significant for the classification.

Just as one classifies conic sections and quadratic forms into parabolic, hyperbolic, and elliptic based on the discriminant $B^2 - 4AC$, the same can be done for a second-order PDE at a given point. 

\underline{Note:} these classifications only apply to second order PDEs. 

If there are $n$ independent variables $x_1, x_2 , \dots, x_n$, a general linear partial differential equation of second order has the form
%
\begin{equation}
Lu = \sum_{i=1}^n \sum_{j=1}^n a_{i,j} \frac{\partial^2 u}{\partial x_i x_j} + \text{ Lower Order Terms } = 0 \nonumber
\end{equation}
%
The classification depends upon the signature of the eigenvalues of the coefficient matrix $a_{i,j}$.
\begin{enumerate}
\item \underline{Elliptic}: The eigenvalues are all positive or all negative.
\item \underline{Parabolic}: The eigenvalues are all positive or all negative, save one that is zero.
\item \underline{Hyperbolic}: There is only one negative eigenvalue and all the rest are positive, or there is only one positive eigenvalue and all the rest are negative.
\item \underline{Ultrahyperbolic}: There is more than one positive eigenvalue and more than one negative eigenvalue, and there are no zero eigenvalues. There is only limited theory for ultrahyperbolic equations (Courant and Hilbert, 1962).
\end{enumerate}

%-------------------------------------------------------------
%-------------------------------------------------------------
\section{Vector Review}

A real $N$-dimensional vector $\bar{x}$ is an ordered set of $N$ real numbers:
%
\begin{equation}
\bar{x} = (x_1, x_2, \dots, x_N) \nonumber
\end{equation}

\textbf{Properties}:
%
\begin{enumerate}
\item sum: $\bar{x} + \bar{y} = (x_1 + y_1, x_2 + y_2, \dots, x_N + y_N)$
\item scalar multiple: $c\bar{x} = (cx_1, cx_2, \dots, cx_N)$
\item dot product: $\bar{x} \cdot \bar{y} = (x_1 y_1, x_2 y_2, \dots, x_N y_N)$
\item Euclidean norm (length): $||\bar{x}|| = (x_1^2 + x_2^2 + \dots + x_N^2)^{1/2}$
\item $||\bar{x}||^2 = \bar{x} \cdot \bar{x}$
\item distance from $\bar{x}$ to $\bar{y}$: $||\bar{x} - \bar{y}|| = ((x_1 - y_1)^2 + (x_2 - y_2)^2 + \dots + (x_N - y_N)^2)^{1/2}$
\item commutative property: $\bar{x} + \bar{y} = \bar{y} + \bar{x}$
\item associative property: $(\bar{x} + \bar{y}) + \bar{z} = \bar{x} + (\bar{y} + \bar{z})$
\item distributive property: $a(\bar{x} + \bar{y}) = a\bar{x} + a\bar{y}$
\end{enumerate}


%-------------------------------------------------------------
%-------------------------------------------------------------
\section{Matrices}

\begin{align}
    \ve{A} &= [a_{ij}]_{M\times N}   =    \begin{pmatrix}
      a_{11} & a_{12} & \cdots & a_{1j} & \cdots & a_{1N} \\
      a_{21} & a_{22} & \cdots & a_{2j} & \cdots & a_{2N} \\
       \vdots & \vdots & \ddots & \vdots & \vdots   & \vdots \\     
      a_{31} & a_{32} & \cdots & a_{ij} & \cdots & a_{iN} \\
      \vdots & \vdots & \vdots & \vdots & \ddots   & \vdots \\
      a_{M1} & a_{M2} & \cdots & a_{Mj} & \cdots & a_{MN} \\
    \end{pmatrix} \nonumber   
\end{align} 
%
where $i = 1, \dots, M$ is the row index and $j = 1, \dots, N$ is the column index.

$\ve{A} \in \mathcal{R}^{M \times N}$ is an $M \times N$ real matrix\\
$\ve{A} \in \mathcal{C}^{M \times N}$ is an $M \times N$ complex matrix

\subsection{Definitions}

Given $\ve{A} \in \mathcal{C}^{M \times N}$, \ve{A} is
%
\begin{enumerate}
\item Symmetric if $\ve{A} = \ve{A}^T$; to create $\ve{A}^T$ $a_{ij} \rightarrow a_{ji}$
\item 
\end{enumerate}


\end{document}